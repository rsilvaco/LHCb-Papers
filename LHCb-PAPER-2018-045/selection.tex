\section{Event selection}
\label{sec:selection}

The selection requirements follow closely those used for the determination
of the branching fractions of the \BdstoKshhp decays, reported in
Ref.~\cite{LHCb-PAPER-2017-010}.
A brief summary of the requirements follows, with emphasis placed on
where they differ from those used in the branching fraction analysis.

Decays of \decay{\KS}{\pip\pim} are reconstructed in two different categories:
the first involving \KS mesons that decay early enough for the
resulting pions to be reconstructed in the \velo; and the
second containing those \KS mesons that decay later, such that track
segments from the pions cannot be formed in the \velo.
These categories are referred to as \emph{\LL} and \emph{\DD}, respectively.
While the \LL\ category has better mass, momentum and vertex resolution,
there are approximately twice as many \KS candidates reconstructed in the
\DD\ category.
In the following, \B candidates reconstructed from either a \LL\ or \DD\
\KS candidate, in addition to two oppositely charged tracks, are also
referred to with these category names.
In order to account for changes in the trigger efficiency for each of the
\KS reconstruction categories during the data taking, the data sample is
subdivided into 2011, 2012a, and 2012b data-taking periods.
The 2012b sample is the largest, corresponding to 1.4\invfb, and also has
the highest trigger efficiency.

To suppress backgrounds, in particular combinatorial background formed by
random combinations of unrelated tracks, the events satisfying the trigger
requirements are filtered by a loose preselection, followed by an optimised
multivariate selection.
The selection is designed in such a way to minimise correlation of the
signal efficiency with position in the Dalitz plot, resulting in better
control of the corresponding systematic uncertainties.
Consequenty, the selection exploits heavily the topological features
that arise from the detached vertex of the \B candidate and relies very
little on the kinematics of its decay products.

The preselection of \KS and \B candidates and the training of the
multivariate classifiers, based on a boosted decision tree~(BDT)
algorithm~\cite{Breiman,AdaBoost}, is identical to that reported in
Ref.~\cite{LHCb-PAPER-2017-010}.
The selection requirement placed on the output of the BDTs is independently
optimised for each data sample using a different figure of merit from that in Ref.~\cite{LHCb-PAPER-2017-010}.
A study was performed with pseudoexperiments that were generated using a model containing a set of resonances that might contribute to the \BstoKsKPi Dalitz plot, and signal and background yields corresponding to various possible selection requirements on the BDT output.
The statistical uncertainty on each of the magnitudes and phases of the
resonances in the model as well as the systematic uncertainty
corresponding to the knowledge of the Dalitz plot distribution of the
backgrounds were determined for each selection requirement.
The responses of several figures of merit were compared with the results of
this study and that which showed the closest correspondence to minimising
the uncertainties on the isobar parameters was chosen.
This figure of merit is
%
\begin{equation}
{\cal Q} \equiv \frac{N^2_{\rm sig}}{\left(N_{\rm sig}+N_{\rm bg}\right)^\frac{3}{2}} \,,
\end{equation}
%
where $N_{\rm sig}$ ($N_{\rm bg}$) represents the number of expected signal
(combinatorial background) events, for a given selection, in the signal region
defined as the invariant-mass window of five times the typical resolution
around the \Bd or \Bs mass.
The value of $N_{\rm sig}$ is estimated based on the known branching
fractions and efficiencies, while $N_{\rm bg}$ is calculated by fitting the
sideband above the signal region and extrapolating into the signal region.
It may be noted that ${\cal Q}$ is equal to the product of two other figures of merit considered in the literature: $N_{\rm sig}/\sqrt{N_{\rm sig}+N_{\rm bg}}$ (sometimes referred to as {\it significance}) and $N_{\rm sig}/\left(N_{\rm sig}+N_{\rm bg}\right)$ ({\it purity}).

Particle identification (PID) information is used to assign
each candidate exclusively to one out of four possible final states:
\KsPiPi, \KsKpPim, \KsKmPip, and \KsKK.
The PID requirements are optimised to reduce the cross-feed between the
different signal decay modes using the same figure of merit ${\cal Q}$ introduced for the BDT optimisation. 
Additional PID requirements are applied in order to reduce backgrounds from
decays such as \LbtoKsPip, where the proton is misidentified as a kaon.

Fully reconstructed \B-meson decays into two-body $\DorDsm\hadp$ or
$(\cquark\cquarkbar)\KS$ combinations, where $(\cquark\cquarkbar)$
indicates a charmonium resonance, may result in a \Kshhp final state that
satisfies the selection criteria and has the same \B-candidate
invariant mass distribution as the signal candidates.
The decays of \Lbbar\ baryons to $\Lcbar\hadp$ with \decay{\Lcbar}{\antiproton\KS} also
peak under the signal when the antiproton is misidentified.
A series of invariant mass vetoes, identical to those used in
Ref.~\cite{LHCb-PAPER-2017-010}, are employed to remove these backgrounds.

The fraction of selected events containing more than one \B candidate is
below the level of 1\%.
The candidate to be retained in each event is chosen randomly but in such a
way that the choice is reproducible.


