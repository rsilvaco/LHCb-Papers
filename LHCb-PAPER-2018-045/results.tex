\section{Results}
\label{sec:results}

The flavour-averaged fit fractions are converted into product branching fractions using Eq.~(\ref{eq:Chap-FFhat}) and 
${\cal B}(\BsToKzBarOptKpi) = (84.3 \pm 3.5 \pm 7.4 \pm 3.4)\times10^{-6}$~\cite{LHCb-PAPER-2017-010}, to obtain
\begin{equation*}
\begin{array}{rcl}      
\Br{\Bs \to \KstarIpm\Kmp; \KstarIpm \to \KorKbarz\pipm}               &=& \\
& \multicolumn{2}{l}{\hspace{-4mm}(   12.4 \pm   0.8   \pm   0.5   \pm   \phz2.7   \pm   1.3)  \times 10^{-6} \,,} \\  
\Br{\Bs \to \KpiSpm\Kmp}                                               &=& \\
& \multicolumn{2}{l}{\hspace{-4mm}(   24.9 \pm   1.8   \pm   0.5   \pm      20.0   \pm   2.6)  \times 10^{-6} \,,} \\
\Br{\Bs \to \KstarIIIpm\Kmp; \KstarIIIpm \to \KorKbarz\pipm}           &=& \\
& \multicolumn{2}{l}{\hspace{-4mm}(\phz3.4 \pm   0.8   \pm   0.4   \pm   \phz5.4   \pm   0.4)  \times 10^{-6} \,,} \\
\Br{\Bs \to \KstarIzoptbar\KorKbarz; \KstarIzoptbar \to \Kmp\pipm}     &=& \\
& \multicolumn{2}{l}{\hspace{-4mm}(   13.2 \pm   1.9   \pm   0.8   \pm   \phz2.9   \pm   1.4)  \times 10^{-6} \,,} \\
\Br{\Bs \to \KpiSz\KorKbarz}                                           &=& \\
& \multicolumn{2}{l}{\hspace{-4mm}(   26.2 \pm   2.0   \pm   0.7   \pm   \phz7.3   \pm   2.8)  \times 10^{-6} \,,} \\
\Br{\Bs \to \KstarIIIzoptbar\KorKbarz; \KstarIIIzoptbar \to \Kmp\pipm} &=& \\
& \multicolumn{2}{l}{\hspace{-4mm}(\phz5.6 \pm   1.5   \pm   0.6   \pm   \phz7.0   \pm   0.6)  \times 10^{-6} \,,}
\end{array}  
\end{equation*}
%-------------------------------------------------------------------------------
where the uncertainties are respectively statistical, systematic related to experimental and model uncertainties, and due to
the uncertainty on ${\cal{B}}(\BsToKzBarOptKpi)$.\footnote{The notation \KpiS\ indicates the total \kpi\ S-wave that is modelled by the LASS lineshape.}

It is possible to use the composition of the LASS lineshape to obtain separately the fractions of the contributing parts.
Integrating separately the resonant part, the effective range part, and the coherent sum, for both the \KpiSz and the \KpiSpm components,
the \KstarIIpm or \KstarIIzoptbar resonances are found to account for 78\%, the effective range term 46\%, and destructive interference between the two terms is responsible for the excess 24\%.
The branching fractions of the two nonresonant parts are found to be 
\begin{eqnarray*}
\Br{\Bs \to \KpiNRpm\Kmp}     &=& (11.4   \pm   0.8   \pm   0.2   \pm   9.2   \pm   1.2   \pm   0.5)  \times 10^{-6} \,, \\
\Br{\Bs \to \KpiNRz\KorKbarz} &=& (12.1   \pm   0.9   \pm   0.3   \pm   3.3   \pm   1.3   \pm   0.5)  \times 10^{-6} \,,
\end{eqnarray*}
where the fifth error is due to the uncertainty on the proportion of the \KpiS\ component due to the effective range part.
Similarly, the product branching fractions for the \KstarII\ resonances are
\begin{equation*}
\begin{array}{rcl}      
\Br{\Bs \to \KstarIIpm\Kmp; \KstarIIpm \to \KorKbarz\pipm}           &=& \\
& \multicolumn{2}{l}{\hspace{-14mm}(19.4   \pm   1.4   \pm   0.4   \pm    15.6   \pm   2.0   \pm   0.3)  \times 10^{-6} \,,} \\
\Br{\Bs \to \KstarIIzoptbar\KorKbarz; \KstarIIzoptbar \to \Kmp\pipm} &=& \\
& \multicolumn{2}{l}{\hspace{-14mm}(20.5   \pm   1.6   \pm   0.6   \pm \phz5.7   \pm   2.2   \pm   0.3)  \times 10^{-6} \,.}
\end{array}  
\end{equation*}
% where the fifth error is due to the uncertainty on the proportion of the \KpiS component due to the \KstarII resonance.

Results for the various \Kstar resonances are further corrected by their branching fractions to \kpi to obtain the quasi-two-body branching fractions.
The branching fractions to \kpi are~\cite{PDG2017}: $\Br{\KstarI \to \kpi} = 100\%$, $\Br{\KstarII \to \kpi} = (93 \pm 10)\%$ and $\Br{\KstarIII \to \kpi} = (49.9 \pm 1.2)\%$.
In addition, the values of $\Br{K^{*} \to K \pi}$ are scaled by the corresponding squared Clebsch-Gordan coefficients, 
\ie\ $2/3$ for both $\KstarzorKstarzb \to \Kpm \pimp$ and $\Kstarpm \to \KorKbarz \pipm$. 
The branching fractions are thus
%-------------------------------------------------------------------------------
\begin{eqnarray*}      
\Br{\Bs\to \KstarIpm\Kmp}              &=& (18.6      \pm   1.2   \pm   0.8   \pm   \phz4.0   \pm   2.0)  \times 10^{-6} \,, \\
\Br{\Bs\to \KstarIIpm\Kmp}             &=& (31.3      \pm   2.3   \pm   0.7   \pm      25.1   \pm   3.3)  \times 10^{-6} \,, \\
\Br{\Bs\to \KstarIIIpm\Kmp}            &=& (10.3      \pm   2.5   \pm   1.1   \pm      16.3   \pm   1.1)  \times 10^{-6} \,, \\
\Br{\Bs \to \KstarIzoptbar\KorKbarz}   &=& (19.8      \pm   2.8   \pm   1.2   \pm   \phz4.4   \pm   2.1)  \times 10^{-6} \,, \\
\Br{\Bs \to \KstarIIzoptbar\KorKbarz}  &=& (33.0      \pm   2.5   \pm   0.9   \pm   \phz9.1   \pm   3.5)  \times 10^{-6} \,, \\
\Br{\Bs \to \KstarIIIzoptbar\KorKbarz} &=& (16.8      \pm   4.5   \pm   1.7   \pm      21.2   \pm   1.8)  \times 10^{-6} \,,
\end{eqnarray*}  
%-------------------------------------------------------------------------------
where the uncertainties are respectively statistical, systematic related to experimental and model uncertainties, and due to
the uncertainty on ${\cal{B}}(\BsToKzBarOptKpi)$, $\Br{\Kstar\to\kpi}$ and,
in the case of \KstarII, the uncertainty of the proportion of the \KpiS
component due to the \KstarII resonance.

%The measurements of the previously observed decay modes
%$\Bs\to\KstarIpm\Kmp$ and $\Bs\to\KstarIzoptbar\KorKbarz$ are somewhat
%larger than the results reported in Refs.~\cite{LHCb-PAPER-2014-043}
%and~\cite{LHCb-PAPER-2015-018}, namely
%\begin{eqnarray}
%\Br{\Bs \to \KstarIpm\Kmp}           &=& \left( 12.7 \pm 1.9 \pm 1.9 \right) \times 10^{-6} \, , \\
%\Br{\Bs \to \KstarIzoptbar\KorKbarz} &=& \left( 10.9 \pm 2.5 \pm 1.2 \right) \times 10^{-6} \, .
%\end{eqnarray}
%Partly this is due to the increased BF of \BstoKsKPi from the updated
%analysis using both 2011 and 2012 data.
%Moreover, there are many improvements that have been brought to this analysis, starting from a new Stripping and 
%reconstruction to a more comprehensive signal model.
%Finally, this amplitude analysis can better separate the \KstarI states from the other contributions in the Dalitz plot, 
%in particular the S-wave.
%Therefore, the obtained results are meant to supersede the previous published measurement. 

