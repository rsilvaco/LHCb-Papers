\section{Dalitz plot analysis formalism}
\label{sec:formalism}

The Dalitz plot~\cite{Dalitz:1953cp} describes the phase-space of a three-body decay in terms of two of the three possible two-body invariant mass squared combinations.
In $\Bs \to \KS\Kpm\pimp$ decays the most significant resonant structures are expected to be from excited kaon states decaying to $\KS\pimp$ or $\Kpm\pimp$ and therefore these are used to define the Dalitz plot axes.
For a fixed $\Bs$ mass, these two invariant mass squared combinations can be used to calculate all other relevant kinematic quantities.

The Dalitz plot analysis involves developing a model that describes the variation of the complex decay amplitudes over the full phase-space of a three-body decay.
The distribution of decays seen in experiment is related to the square of the magnitude of the amplitude, modified to account for detection efficiency and background contributions.  
As described in Sec.~\ref{sec:Introduction}, this is only an approximation for $\Bs \to \KS\Kpm\pimp$ decays, where the physical distribution in each final state depends on the incoherent sum of two contributions.  
A single amplitude is nonetheless used to model the data, since it is not possible to separate the two contributing amplitudes without initial state flavour tagging; a systematic uncertainty is assigned to account for possible biases induced by this approximation.
The Dalitz plot fit is performed using the {\sc Laura++}~\cite{Laura++} package, with the different final states, \KS reconstruction categories and data-taking periods handled using the {\it J}{\sc fit} method~\cite{Ben-Haim:2014afa}.

The isobar model~\cite{Fleming:1964zz,Morgan:1968zza,Herndon:1973yn} is used to describe the complex decay amplitude. 
The total amplitude is given by the coherent sum of amplitudes from $N$ intermediate contributions,
% and is given by
\begin{equation}\label{eqn:amp}
  {\cal A}\left(m^2(\KS\pimp), m^2(\Kpm\pimp)\right) = \sum_{j=1}^{N} c_j F_j\left(m^2(\KS\pimp), m^2(\Kpm\pimp)\right) \,,
\end{equation}
where $c_j$ are complex coefficients describing the relative strength of each intermediate process. 
The resonant dynamics are contained in the $F_j\left(m^2(\KS\pimp),m^2(\Kpm\pimp)\right)$ terms, which are normalised such that the integral of the squared magnitude over the Dalitz plot is unity for each term.
For a $\KS\pimp$ resonance $F_j\left(m^2(\KS\pimp),m^2(\Kpm\pimp)\right)$ is given by
\begin{equation}
  \label{eq:ResDynEqn}
  F\left(m^2(\KS\pimp), m^2(\Kpm\pimp)\right) = 
  R\left(m(\KS\pimp)\right) \times X(|\vec{p}\,|\,r_{\rm BW}) \times X(|\vec{q}\,|\,r_{\rm BW}) 
  \times T(\vec{p},\vec{q}\,) \, ,
\end{equation}
where $\vec{p}$ is the bachelor particle\footnote{
  The ``bachelor'' particle is that not forming the resonance, \ie\ the $\Kpm$ in this example.} momentum and $\vec{q}$ is the momentum of one of the resonance daughters, both evaluated in the $\KS\pimp$ rest frame.
The $R$ functions are the mass lineshapes, typically described by the relativistic Breit--Wigner function with alternative shapes used in some specific situations.
The $X$ and $T$ terms describe barrier factors and angular distributions, respectively, and depend on the orbital angular momentum between the resonance and the bachelor particle, $L$;
the barrier factors $X$ are evaluated in terms of the Blatt--Weisskopf radius parameter $r_{\rm BW}$ for which a default value of $4.0 \gev^{-1}\hbar c$ is used.
The angular distributions are given in the Zemach tensor formalism~\cite{Zemach:1963bc,Zemach:1968zz}, and are proportional to the Legendre polynomials, $P_L(x)$, where $x$ is the cosine of the angle between $\vec{p}$ and $\vec{q}$ (referred to as the helicity angle). 
Detailed expressions for the functions $R$, $X$ and $T$ can be found in Ref.~\cite{Laura++}.

% The $X(z)$ terms are Blatt--Weisskopf barrier factors~\cite{blatt-weisskopf}, where $z=|\vec{q}\,|\,r_{\rm BW}$ or $|\vec{p}\,|\,r_{\rm BW}$ and $r_{\rm BW}$ is the barrier radius which is set to $4.0\gev^{-1}\approx 0.8\fm$~\cite{LHCb-PAPER-2014-036} for all resonances. 
% The barrier factors are angular momentum dependent and are given by
% \begin{equation}\begin{array}{rcl}
% L = 0 \ : \ X(z) & = & 1\,, \\
% L = 1 \ : \ X(z) & = & \sqrt{\frac{1 + z_0^2}{1 + z^2}}\,, \\
% L = 2 \ : \ X(z) & = & \sqrt{\frac{z_0^4 + 3z_0^2 + 9}{z^4 + 3z^2 + 9}}\,,\\
% L = 3 \ : \ X(z) & = & \sqrt{\frac{z_0^6 + 6z_0^4 + 45z_0^2 + 225}{z^6 + 6z^4 + 45z^2 + 225}}\,,
% \end{array}\label{eq:BWFormFactors}\end{equation}
% where $z_0$ is the value of $z$ at the pole mass of the resonance and $L$ is the orbital angular momentum between the resonance and the bachelor particle.
% Since the parent and daughter particles all have zero spin, $L$ is also the spin of the resonance.

% The $T(\vec{p},\vec{q})$ terms describe the angular distributions in the Zemach tensor formalism~\cite{Zemach:1963bc,Zemach:1968zz} and are given by
% \begin{equation}\begin{array}{rcl}
% L = 0 \ : \ T(\vec{p},\vec{q}) & = & 1\,,\\
% L = 1 \ : \ T(\vec{p},\vec{q}) & = & -\,2\,\vec{p}\cdot\vec{q}\,,\\
% L = 2 \ : \ T(\vec{p},\vec{q}) & = & \frac{4}{3} \left[3(\vec{p}\cdot\vec{q}\,)^2 - (|\vec{p}\,||\vec{q}\,|)^2\right]\,,\\
% L = 3 \ : \ T(\vec{p},\vec{q}) & = & -\,\frac{24}{15} \left[5(\vec{p}\cdot\vec{q}\,)^3 - 3(\vec{p}\cdot\vec{q}\,)(|\vec{p}\,||\vec{q}\,|)^2\right]\,.
% \end{array}\label{eq:ZTFactors}\end{equation}
% These expressions are proportional to the Legendre polynomials, $P_L(x)$, where $x$ is the cosine of the angle between $\vec{p}$ and $\vec{q}$ (referred to as the helicity angle). 

% The $R\left(m(\KS\pimp)\right)$ functions are the mass lineshapes.
% Resonant contributions are typically described by the relativistic Breit--Wigner (RBW) function
% \begin{equation}
% \label{eq:RelBWEqn}
% R(m) = \frac{1}{(m_0^2 - m^2) - i\, m_0 \Gamma(m)} \,,
% \end{equation}
% where the mass-dependent decay width is
% \begin{equation}
% \label{eq:GammaEqn}
% \Gamma(m) = \Gamma_0 \left(\frac{q}{q_0}\right)^{2L+1}
% \left(\frac{m_0}{m}\right) X^2(q\,r_{\rm BW}) \,,
% \end{equation}
% where $q_0$ is the value of $q = |\vec{q}\,|$ when the invariant mass is equal to the pole mass of the resonance, $m_0$.

% The large phase space available in \B decays allows for the presence of nonresonant amplitudes (\ie\ contributions that do not proceed via a known resonance) that vary across the Dalitz plot.
% An exponential form factor (EFF) has been found to describe nonresonant contributions well in several Dalitz plot analyses of \B decays~\cite{Garmash:2004wa}, 
% \begin{equation}
%   R(m) = \exp\left[-\alpha m^2\right] \, ,
%   \label{eq:nonres}
% \end{equation}
% where $\alpha$ is a shape parameter that must be determined from the data and $m$ is a two-body invariant mass ($m(\KS\pimp)$ in this example).

% The RBW function is a good model for narrow resonances that are well separated from any other resonant or nonresonant
% contribution of the same spin.
% This approach is known to break down in the $K\pi$ S-wave because the $\KstarIIb$ resonance interferes strongly with a slowly varying nonresonant term, as described in Ref.~\cite{Meadows:2007jm}.
% The LASS lineshape~\cite{lass} has been developed to combine these two contributions,
% \begin{eqnarray}
%  \label{eq:LASSEqn}
%   R(m) & = & \frac{m}{q \cot{\delta_B} - iq} + \exp\left[2i \delta_B\right]
%   \frac{m_0 \Gamma_0 \frac{m_0}{q_0}}
%        {(m_0^2 - m^2) - i m_0 \Gamma_0 \frac{q}{m} \frac{m_0}{q_0}}\, , \\
% {\rm where} \ \cot{\delta_B} & = & \frac{1}{aq} + \frac{1}{2} r q \, ,
% \end{eqnarray}
% and where $m_0$ and $\Gamma_0$ are the pole mass and width of the $\KstarIIb$ state, and $a$ and $r$ are shape parameters.

The complex coefficients $c_j$, defined in Eq.~(\ref{eqn:amp}), are determined from the fit to data.
These are used to obtain fit fractions for each component $j$, which provide a robust and convention-independent way to report the results of the analysis.
The fit fractions are defined as the integral over one Dalitz plot ($\KS\Kp\pim$ or $\KS\Km\pip$) of the amplitude for each intermediate component squared, divided by that of the coherent matrix element squared for all intermediate contributions, 
\begin{equation}
{\it FF}_j =
\frac
{\int\!\!\int_{\rm DP}\left|c_j F_j\right|^2~dm^2(\KS\pimp)\,dm^2(\Kpm\pimp)}
{\int\!\!\int_{\rm DP}\left|{\cal A}\right|^2~dm^2(\KS\pimp)\,dm^2(\Kpm\pimp)} \, ,
\label{eq:fitfraction}
\end{equation}
where the dependence of $F_j$ and ${\cal A}$ on Dalitz plot position has been omitted for brevity.
The fit fractions need not sum to unity due to possible net constructive or destructive interference.
% , described by interference fit fractions defined by 
% \begin{equation}
%   {\it FF}_{ij} =
%   \frac
%   {\int\!\!\int_{\rm DP} 2 \, \Real\left[c_ic_j^* F_iF_j^*\right]~dm^2(\KS\pimp)\,dm^2(\Kpm\pimp)}
%   {\int\!\!\int_{\rm DP}\left|{\cal A}\right|^2~dm^2(\KS\pimp)\,dm^2(\Kpm\pimp)} \, ,
%   \label{eq:intfitfraction}
% \end{equation}
% where the dependence of $F_i^{(*)}$ and ${\cal A}$ on the Dalitz plot position has been omitted.
% By definition, the sum of all fit fractions and interference fit fractions
% with $i<j$ only is equal to unity.  

%% For the purpose of the branching ratio calculation, it is useful to consider quantities averaged over the two different final states.
For this analysis, it is useful to define also flavour-averaged fit fractions $\widehat{FF}_{j}$, where the numerator and denominator of Eq.~(\ref{eq:fitfraction}) are replaced by sums of the same quantities over both final states, and it is understood that a resonance corresponding to $j$ in one Dalitz plot will be replaced by its conjugate in the other (\eg\ \KstarIm\ in the $\KS\Kp\pim$ final state and \KstarIp\ for $\KS\Km\pip$).
These can be converted into product of branching fractions for the \Bs\ and \Kstar\ decays by multiplying by the known $\BsToKzBarOptKpi$ branching fraction,
\begin{equation}
  \label{eq:Chap-FFhat}
  \Br{\Bs \to \Kstar\kaon; \Kstar\to\kaon\pion} = \widehat{FF}_{j} \times \Br{\BsToKzBarOptKpi} \,, 
\end{equation}  
where $\Br{\Bs \to \Kstar\kaon}$ is the sum of the branching fractions for the two conjugate final states.
