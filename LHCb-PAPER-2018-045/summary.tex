\section{Summary}
\label{sec:summary}

In summary, the first amplitude analysis of $\Bs \to \KS\Kpm\pimp$ decays has been presented, using a data sample corresponding to $3.0\invfb$ of $pp$ collision data collected 
by the LHCb experiment. 
A good description of the data is obtained with a model containing contributions 
from both neutral and charged resonant states $\Kstar(892)$, $K^*_0(1430)$ and $K^*_0(1430)$. 
Measurements of the previously observed decay modes
$\Bs\to\KstarIpm\Kmp$ and $\Bs\to\KstarIzoptbar\KorKbarz$ are consistent with theoretical predictions~\cite{Cheng:2014uga,Li:2014fla,Li:2018qrm}, and also consistent with but larger than the previous LHCb results~\cite{LHCb-PAPER-2014-043,LHCb-PAPER-2015-018}, which they supersede.
This is attributed partly due to the larger \BstoKsKPi\ branching fraction determined in the updated analysis based on both 2011 and 2012 data~\cite{LHCb-PAPER-2017-010} compared to its previous determination~\cite{LHCb-PAPER-2013-042}. 
% and from improvements in the selection.
This amplitude analysis provides better separation of the \KstarI\ states from the other contributions in the Dalitz plot, in particular the S-wave, and more accurate estimation of associated systematic uncertainties.
% Therefore, the obtained results are meant to supersede the previous published measurement. 
Contributions from $K^*_0(1430)$ states are observed for the first time with significance above $10$ standard deviations. 

Increases in the data sample size will allow the reduction of both statistical and systematic uncertainties on these results.
As significantly larger samples are anticipated following the upgrade of LHCb~\cite{LHCb-TDR-012,LHCb-PII-EoI}, it will be possible to extend the analysis to include flavour tagging and decay-time-dependence and therefore to obtain sensitivity to test the SM through measurement of \CP\ violation parameters in $\Bs \to \KS\Kpm\pimp$ decays.



