%\pdfoutput=1
% Uncomment line above if submitting to arXiv and using pdflatex

% $Id: main.tex 112040 2017-09-01 06:50:44Z fwilson $
% ============================================================================
% Purpose: Template for LHCb documents
% Authors: Tomasz Skwarnicki, Roger Forty, Ulrik Egede
% Created on: 2010-09-24
% ============================================================================
\documentclass[12pt,a4paper]{article}
%%\documentclass[12pt,letter]{article}
% For two column text, add "twocolumn" as an option to the document
% class. Also uncomment the two "onecolumn" and "twocolumn" lines
% around the title page below.

% Variables that controls behaviour
\usepackage{ifthen} % for conditional statements
\newboolean{pdflatex}
\setboolean{pdflatex}{true} % False for eps figures 

\newboolean{articletitles}
\setboolean{articletitles}{true} % False removes titles in references

\newboolean{uprightparticles}
\setboolean{uprightparticles}{false} %True for upright particle symbols

\newboolean{inbibliography}
\setboolean{inbibliography}{false} %True once you enter the bibliography

% Define titles and authors here. It will then be used both in metadata and in
% what is printed on the front page.
\def\paperauthors{LHCb collaboration} % Leave as is for PAPER and CONF
\def\paperasciititle{Amplitude analysis of Bs0 -> KS0 K+- pi-+ decays} % Set ASCII title here
\def\papertitle{Amplitude analysis of $\Bs \to \KS\Kpm\pimp$ decays} % Latex formatted title
\def\paperkeywords{{High Energy Physics}, {LHCb}} % Comma separated list
%\def\papercopyright{CERN on behalf of the LHCb collaboration}
\def\papercopyright{\the\year\ CERN for the benefit of the LHCb collaboration} % new since 9/Apr/2018
\def\paperlicence{CC-BY-4.0 licence}
\def\paperlicenceurl{https://creativecommons.org/licenses/by/4.0/}

\input{preamble}
%\usepackage{longtable} % only for template; not usually to be used in PAPERs
%%%Add comments with at least three %%% preceding.
%%% Add new sections with one % preceding

%%% Add new subsections with two %% preceding
% 

%%%%%%%%%%%%%
% Non standard particles
%%%%%%%%%%%%%
% light vectors
\ifthenelse{\boolean{uprightparticles}}%
{\def\Prho      {\ensuremath{\uprho}\xspace}
 \def\Pphi      {\ensuremath{\upphi}\xspace}
}
{\def\Prho      {\ensuremath{\rho}\xspace}
 \def\Pphi      {\ensuremath{\phi}\xspace}
}
\def\rhoz   {\ensuremath{\Prho^0}\xspace}
\def\rhozs  {\ensuremath{\Prho^0\mbox\,\rm{s}}\xspace}
\def\rhop   {\ensuremath{\Prho^+}\xspace}
\def\rhom   {\ensuremath{\Prho^-}\xspace}
%\def\phiz   {\ensuremath{\Pphi^0}\xspace}
%\def\phizs  {\ensuremath{\Pphi^0\mbox\,\rm{s}}\xspace}
\def\had  {\ensuremath{\Ph}\xspace}
\def\hadp  {\ensuremath{\Ph^+}\xspace}
\def\hadm  {\ensuremath{\Ph^-}\xspace}
\def\hadpm  {\ensuremath{\Ph^{\pm}}\xspace}
\def\hadmp  {\ensuremath{\Ph^{\mp}}\xspace}
\def\hadprim  {\ensuremath{\had^{\prime}}\xspace}
\def\hadprimp  {\ensuremath{\had^{\prime+}}\xspace}
\def\hadprimm {\ensuremath{\had^{\prime-}}\xspace}
\def\hadprimpm  {\ensuremath{\had^{\prime\pm}}\xspace}
\def\hadprimmp  {\ensuremath{\had^{\prime\mp}}\xspace}

\def\BdorBdbar    {\ensuremath{\kern 0.18em\optbar{\kern -0.18em B}{}^0}\xspace}
\def\BsorBsbar    {\ensuremath{\kern 0.18em\optbar{\kern  0.06em B_s}{}^0}\xspace}
%\def\BdsorBdsbar    {\ensuremath{\kern 0.18em\optbar{\kern  0.72em B_{d,s}}{}^0}\xspace}

\def\pipi  {\ensuremath{\pion^+\pion^-}\xspace}

\def\kpi       {\ensuremath{\kaon\pion}\xspace}
\def\kk      {\ensuremath{\Kp\Km}\xspace}
\def\kppim     {\ensuremath{\Kp\pim}\xspace}
\def\kmppipm   {\ensuremath{\Kmp\pipm}\xspace}
\def\kmppipmpiz   {\ensuremath{\Kmp\pipm\piz}\xspace}
\def\kmppipmpipmpimp {\ensuremath{\Kmp\pipm\pipm\pimp}\xspace}
\def\kmpip     {\ensuremath{\Km\pip}\xspace}

% now defined in lhcb-symbols-def
%\def\KorKbarz {\ensuremath{\KorKbar^0}\xspace}

\def\KstarI        {\ensuremath{\Kstar(892)}\xspace}
\def\KstarIp       {\ensuremath{\Kstar(892)^+}\xspace}
\def\KstarIm       {\ensuremath{\Kstar(892)^-}\xspace}
\def\KstarIpm      {\ensuremath{\Kstar(892)^{\pm}}\xspace}
\def\KstarIz       {\ensuremath{\Kstar(892)^0}\xspace}
\def\KstarIzb      {\ensuremath{\Kstarb(892)^0}\xspace}
\def\KstarIzoptbar {\ensuremath{\KorKbar\!^*(892)^0}\xspace}

\def\KstarII        {\ensuremath{\kaon^*_0(1430)}\xspace}
\def\KstarIIb       {\ensuremath{\Kbar{}^*_0(1430)}\xspace}
\def\KstarIIp       {\ensuremath{\kaon^*_0(1430)^+}\xspace}
\def\KstarIIm       {\ensuremath{\kaon^*_0(1430)^-}\xspace}
\def\KstarIIpm      {\ensuremath{\kaon^*_0(1430)^{\pm}}\xspace}
\def\KstarIIz       {\ensuremath{\kaon^*_0(1430)^0}\xspace}
\def\KstarIIzb      {\ensuremath{\Kbar{}^*_0(1430)^0}\xspace}
\def\KstarIIzoptbar {\ensuremath{\KorKbar\!^*_0(1430)^0}\xspace}

\def\KstarIII        {\ensuremath{\kaon^*_2(1430)}\xspace}
\def\KstarIIIp       {\ensuremath{\kaon^*_2(1430)^+}\xspace}
\def\KstarIIIm       {\ensuremath{\kaon^*_2(1430)^-}\xspace}
\def\KstarIIIpm      {\ensuremath{\kaon^*_2(1430)^{\pm}}\xspace}
\def\KstarIIIz       {\ensuremath{\kaon^*_2(1430)^0}\xspace}
\def\KstarIIIzb      {\ensuremath{\Kbar{}^*_2(1430)^0}\xspace}
\def\KstarIIIzoptbar {\ensuremath{\KorKbar\!^*_2(1430)^0}\xspace}

\def\KpiS   {\ensuremath{(\kaon\pion)^*_0}\xspace}
\def\KpiSz  {\ensuremath{(\Kmp\pipm)^*_0}\xspace}
\def\KpiSpm {\ensuremath{(\KorKbarz\pipm)^*_0}\xspace}

\def\KpiNR   {\ensuremath{(\kaon\pion)_\mathrm{NR}}\xspace}
\def\KpiNRz  {\ensuremath{(\Kmp\pipm)_\mathrm{NR}}\xspace}
\def\KpiNRpm {\ensuremath{(\KorKbarz\pipm)_\mathrm{NR}}\xspace}

\def\pp    {\ensuremath{\proton\proton}\xspace}
\def\ppbar    {\ensuremath{\proton\antiproton}\xspace}

%%%%%%%%%%%%%
% Decays
%%%%%%%%%%%%%
\newcommand{\subdecay}[2]{\ensuremath{#1(\!\to #2)}\xspace}
% B decays (charmless)
\def\BdtoKzKK   {\decay{\Bd}{\Kz \Kp \Km}}
\def\BdtoKzPiPi   {\decay{\Bd}{\Kz \pip \pim}}
\def\BdtoKzKPi   {\decay{\Bd}{\Kz \Kpm \pimp}}

\def\BstoKzKK   {\decay{\Bs}{\Kz \Kp \Km}}
\def\BstoKzPiPi   {\decay{\Bs}{\Kz \pip \pim}}
\def\BstoKzKPi   {\decay{\Bs}{\Kz \Kpm \pimp}}

\def\BdstoKzKK   {\decay{\BdorBs}{\Kz \Kp \Km}}
\def\BdstoKzPiPi   {\decay{\BdorBs}{\Kz \pip \pim}}
\def\BdstoKzKPi   {\decay{\BdorBs}{\Kz \Kpm \pimp}}

\def\BdtoKsKK   {\decay{\Bd}{\KS \Kp \Km}}
\def\BdtoKsPiPi   {\decay{\Bd}{\KS \pip \pim}}
\def\BdtoKsKPi   {\decay{\Bd}{\KS \Kpm \pimp}}
\def\BdtoKsKpPim   {\decay{\Bd}{\KS \Kp \pim}}
\def\BdtoKsPipKm   {\decay{\Bd}{\KS \Km \pip}}

\def\BstoKsKK   {\decay{\Bs}{\KS \Kp \Km}}
\def\BstoKsPiPi   {\decay{\Bs}{\KS \pip \pim}}
\def\BstoKsKPi   {\decay{\Bs}{\KS \Kpm \pimp}}
\def\BstoKsKpPim   {\decay{\Bs}{\KS \Kp \pim}}
\def\BstoKsPipKm   {\decay{\Bs}{\KS \Km \pip}}

\def\BdstoKsKK   {\decay{\BdorBs}{\KS \Kp \Km}}
\def\BdstoKsPiPi   {\decay{\BdorBs}{\KS \pip \pim}}
\def\BdstoKsKPi   {\decay{\BdorBs}{\KS \Kpm \pimp}}

\def\Kshh{\ensuremath{\KS \hadp \hadm}\xspace}
\def\Kshhp{\ensuremath{\KS \hadp \hadprimm}\xspace}
\def\KshhpSS{\ensuremath{\KS \hadp \hadprimp}\xspace}

\def\KsPiPi{\ensuremath{\KS \pip \pim}\xspace}
\def\KsKPi{\ensuremath{\KS \Kpm \pimp}\xspace}
\def\KsPiK{\ensuremath{\KS \pipm \Kmp}\xspace}
\def\KsKK{\ensuremath{\KS \Kp \Km}\xspace}
\def\KsPiP{\ensuremath{\KS \pip \antiproton}\xspace}

\def\KsKpPim{\ensuremath{\KS \Kp \pim}\xspace}
\def\KsKmPip{\ensuremath{\KS \Km \pip}\xspace}
\def\KsPipKm{\ensuremath{\KS \pip \Km}\xspace}
\def\KsPimKp{\ensuremath{\KS \pim \Kp}\xspace}

\def\BdtoKshhp   {\decay{\Bd}{\KS \hadp \hadprimm}}
\def\BdtoKshh   {\decay{\Bd}{\KS \hadp \hadm}}

\def\BstoKshhp   {\decay{\Bs}{\KS \hadp \hadprimm}}
\def\BstoKshh   {\decay{\Bs}{\KS \hadp \hadm}}

\def\BdstoKshhp   {\decay{\BdorBs}{\KS \hadp \hadprimm}}
\def\BdstoKshh   {\decay{\BdorBs}{\KS \hadp \hadm}}

\def\BdtoKPiPiz {\decay{\Bd}{\Kp \pim \piz}}
\def\BstoKPiPiz {\decay{\Bs}{\Km \pip \piz}}

% Additional signal modes specific for the KSKpi DP
\newcommand{\BsToKSKppim}{\decay{\Bs}{\KS\Kp\pim}}
\newcommand{\BsToKSKmpip}{\decay{\Bs}{\KS\Km\pip}}
\newcommand{\BdToKSKppim}{\decay{\Bd}{\KS\Kp\pim}}
\newcommand{\BdToKSKmpip}{\decay{\Bd}{\KS\Km\pip}}
\newcommand{\BdToKSpipi}{\decay{\Bd}{\KS\pip\pim}}
\newcommand{\BsToKSpipi}{\decay{\Bs}{\KS\pip\pim}}
\newcommand{\BdToKSKK}{\decay{\Bd}{\KS\Kp\Km}}
\newcommand{\BsToKSKK}{\decay{\Bs}{\KS\Kp\Km}}
\newcommand{\BdToKSKpi}{\decay{\Bd}{\KS\Kpm\pimp}}
\newcommand{\BsToKSKpi}{\decay{\Bs}{\KS\Kmp\pip}}

% B decays (charmless backgrounds)
\def\BdtoetapKs {\decay{\Bd}{\etapr \KS}}
\def\BdtoetapKstoRhoGamma   {\decay{\Bd}{\etapr (\rhoz \g) \KS}}
\def\BdtoetapKstoetapipi   {\decay{\Bd}{\etapr (\Peta \pipi) \KS}}
\def\BdtoKsPiPiGamma   {\decay{\Bd}{\KS \pip \pim \gamma}}
\def\BdtoKstzrhoztoKsPizPiPi   {\decay{\Bd}{\Kstarz (\KS \piz) \rhoz (\pip \pim)}}
\def\BdtoKstzPhiztoKsPizKK   {\decay{\Bd}{\Kstarz (\KS \piz) \Pphi (\Kp \Km)}}
\def\BstoKstzPhiztoKsPizKK   {\decay{\Bs}{\Kstarz (\KS \piz) \Pphi (\Kp \Km)}}
\def\ButoKstPhiztoKsPiKK   {\decay{\Bu}{\Kstarp (\KS \pip) \Pphi (\Kp \Km)}}
\def\ButoKstPiPitoKsPi   {\decay{\Bu}{\Kstarp (\KS \pip) \pipi}}
\def\BstoKstzKstzbtoKsPizKPi  {\decay{\Bs}{\Kstarz (\KS\piz) \Kstarzb (\Km\pip)}}

% B decays (charmed, charmonia)
\def\BdtoPsisChicKs {\decay{\Bd}{(\jpsi,\psitwos,\chic) \KS}}
\def\BdtoPsiKstoMuMu {\decay{\Bd}{\jpsi(\mup \mun) \KS}}
\def\BdtoPsitwosKs {\decay{\Bd}{\psitwos \KS}}
\def\BdtoPsisKs {\decay{\Bd}{\jpsi(\psitwos) \KS}}
\def\BdtoDmhptoKshm {\decay{\Bd}{\Dm(\KS \hadm) \hadp}}
\def\BstoDsmhptoKshm {\decay{\Bs}{\Dsm(\KS \hadm) \hadp}}
\def\BstoDsmKp {\decay{\Bs}{\Dsm \Kp}}
\def\BstoDmKptoKspimpiz {\decay{\Bs}{\Dm (\KS\pim\piz) \Kp}}
\def\BdtoDmKptoKspimpiz {\decay{\Bd}{\Dm (\KS\pim\piz) \Kp}}
\def\ButoDsth {\decay{\Bu}{\Dstarm \hadp}}
\def\ButoDzKtoKsPiPi {\decay{\Bu}{\Dz(\KS \pip \pim) \Kp}}
\def\ButoDzPitoKsPiPi {\decay{\Bu}{\Dz(\KS \pip \pim) \pip}}
\def\ButoDzPitoKsKK   {\decay{\Bu}{\Dz(\KS \Kp \Km) \pip}}
\def\ButoDzKtoKsKK   {\decay{\Bu}{\Dz(\KS \Kp \Km) \Kp}}
\def\ButoDstKtoKSPiPi   {\decay{\Bu}{\Dstarzb(\Dzb(\KS \pip \pim) \piz ) \Kp}}
\def\ButoDstPitoKSPiPi   {\decay{\Bu}{\Dstarzb(\Dzb(\KS \pip \pim) \piz ) \pip}}
\def\LbtoDsstptoKsPip    {\decay{\Lb}{\Dssm (\Dsm (\KS \Km)\g ) \proton}}
\def\LbtoDsmptoKshm    {\decay{\Lb}{\Dsm(\KS \hadm) \proton}}
\def\LbtoDsmptoKsPim    {\decay{\Lb}{\Dsm(\KS \pim) \proton}}
\def\LbtoLcPitoKsp    {\decay{\Lb}{\Lc(\KS \proton) \pim}}
\def\LbtoKsKp         {\decay{\Lb}{\KS \proton \Km}}
\def\LbtoKsPip         {\decay{\Lb}{\KS \proton \pim}}

% b decays
\def\btou    {\decay{\bquark}{\uquark}}
\def\btoc    {\decay{\bquark}{\cquark}}
\def\btos    {\decay{\bquark}{\squark}}
\def\btod    {\decay{\bquark}{\dquark}}

\def\btoqqbars {\decay{\bquark}{\quark\quarkbar\squark}}
\def\btossbars {\decay{\bquark}{\squark\squarkbar\squark}}
\def\btoccbars {\decay{\bquark}{\cquark\cquarkbar\squark}}

%%%%%%%%%%%%%
% Reconstruciton mode of KS
%%%%%%%%%%%%%
\def\LL   {long\xspace}
\def\DD   {downstream\xspace}

%%%%%%%%%%%%%
% Dalitz variables
%%%%%%%%%%%%%
\newcommand{\mPrime}{\ensuremath{\m^{\prime}}\xspace}
\newcommand{\thPrime}{\ensuremath{\theta^{\prime}}\xspace}



%%%%%%%%%%%%%
% Branching fractions
%%%%%%%%%%%%%
\newcommand{\Br}[1]{\ensuremath{\BF\left(#1\right)}\xspace}

%%%%%%%%%%%%%
% Lumis
%%%%%%%%%%%%%
\newcommand{\intlumfb}[1]{\ensuremath{\intlum{#1}\,\invfb}}
\newcommand{\intlumpb}[1]{\ensuremath{\intlum{#1}\,\invpb}}


%%%%%%%%%%%%%
% Commands
%%%%%%%%%%%%%
\newcommand{\eq}[1]{Eq.~\ref{equation : #1}}
\newcommand{\eqs}[2]{Eqs.~\ref{equation : #1}-\ref{equation : #2}}
\newcommand{\tab}[1]{Table~\ref{tab : #1}}
\newcommand{\tabs}[2]{Tables~\ref{tab : #1}-\ref{tab : #2}}
\newcommand{\fig}[1]{Fig.~\ref{fig : #1}}
\newcommand{\figs}[2]{Figs.~\ref{fig : #1}-\ref{fig : #2}}
\newcommand{\ttst}[1]{\textrm{\scriptsize{#1}}}
\newcommand{\refsec}[1]{Section~\ref{sec : #1}}

\def\phz {\phantom{0}}
\def\pho {\phantom{1}}


%%%%%%%%%%%%%
% Categories
%%%%%%%%%%%%%
\def\TIS   {\texttt{L0TIS}\xspace}
\def\TOS   {\texttt{L0TOSOnly}\xspace}
\def\HltOne   {\texttt{Hlt1TOS}\xspace}
\def\HltOneTwo   {\texttt{Hlt(1,2)TOS}\xspace}

%%%%%%%%%%%%%
% Fit parameters
%%%%%%%%%%%%%
\newcommand{\fitMeanBd}{\ensuremath{\mu_{\Bd}\xspace}}
\newcommand{\fitMeanBs}{\ensuremath{\mu_{\Bs}\xspace}}

%%%%%%%%%%%%%
% math operators
%%%%%%%%%%%%%
\DeclareMathOperator{\erf}{erf}
\DeclareMathOperator{\ierfc}{ierfc}
\renewcommand\Re{\operatorname{Re}}


%%%%%%%%%%%%%
% Packages
%%%%%%%%%%%%%

\def\bender    {\mbox{\textsc{Bender}}\xspace}

\newcommand{\occ}[1]{\multicolumn{1}{c}{#1}}
\newcommand{\tcc}[1]{\multicolumn{2}{c}{#1}}
\newcommand{\thcc}[1]{\multicolumn{3}{c}{#1}}
\newcommand{\fcc}[1]{\multicolumn{4}{c}{#1}}
\newcommand{\fvcc}[1]{\multicolumn{5}{c}{#1}}
\newcommand{\scc}[1]{\multicolumn{6}{c}{#1}}
\newcommand{\sccb}[1]{\multicolumn{6}{c|}{#1}}

\newcommand{\mrthree}[1]{\multirow{3}{*}{#1}}
\newcommand{\mrsix}[1]{\multirow{6}{*}{#1}}

%%%%%%%%%%%%%%%%%%%%%%%%%%
% Easily change the fit model
%%%%%%%%%%%%%%%%%%%%%%%%%%
\def \fitCombShape{Polynomial}
\def \fitCombModel{from5150}

%%%%%%%%%%%%%%%%%%%%%%%%%%%%%%%%%%%%%%%
% Adaptative fit model description
%%%%%%%%%%%%%%%%%%%%%%%%%%%%%%%%%%%%%%%
\def\KSorKSbar      {\ensuremath{\kern 0.12em\optbar{\kern  0.06em K}{}^0}\xspace}
\def\KsorKsbar      {\ensuremath{\kern 0.0em\optbar{\kern  0.06em K}{}^0_{\rm S}}\xspace}
\def\BsToKzBarOptKpi{\decay{\Bs}{\KSorKSbar \Kpm \pimp}}
\def\BdToKzBarOptKpi{\decay{\Bd}{\KSorKSbar \Kpm \pimp}}
\def\BsToKsBarOptKpi{\decay{\Bs}{\KsorKsbar \Kpm \pimp}}

\def\KstarzorKstarzb {{\ensuremath{\KorKbar\!^{*0}}}\xspace}



\begin{document}



%%%%%%%%%%%%%%%%%%%%%%%%%
%%%%% Title     %%%%%%%%%
%%%%%%%%%%%%%%%%%%%%%%%%%
\renewcommand{\thefootnote}{\fnsymbol{footnote}}
\setcounter{footnote}{1}

% $Id: title-LHCb-PAPER.tex 120776 2018-05-29 05:11:17Z pkoppenb $
% ===============================================================================
% Purpose: LHCb-PAPER journal paper title page template
% Author: 
% Created on: 2010-09-25
% ===============================================================================

%%%%%%%%%%%%%%%%%%%%%%%%%
%%%%%  TITLE PAGE  %%%%%%
%%%%%%%%%%%%%%%%%%%%%%%%%
\begin{titlepage}
\pagenumbering{roman}

% Header ---------------------------------------------------
\vspace*{-1.5cm}
\centerline{\large EUROPEAN ORGANIZATION FOR NUCLEAR RESEARCH (CERN)}
\vspace*{1.5cm}
\noindent
\begin{tabular*}{\linewidth}{lc@{\extracolsep{\fill}}r@{\extracolsep{0pt}}}
\ifthenelse{\boolean{pdflatex}}% Logo format choice
{\vspace*{-1.5cm}\mbox{\!\!\!\includegraphics[width=.14\textwidth]{figs/lhcb-logo.pdf}} & &}%
{\vspace*{-1.2cm}\mbox{\!\!\!\includegraphics[width=.12\textwidth]{figs/lhcb-logo.eps}} & &}%
\\
 & & CERN-EP-2018-045 \\  % ID 
 & & LHCb-PAPER-2018-045 \\  % ID 
 & & \today \\ % Date - Can also hardwire e.g.: 23 March 2010
 & & \\
% not in paper \hline
\end{tabular*}

\vspace*{4.0cm}

% Title --------------------------------------------------
{\normalfont\bfseries\boldmath\huge
\begin{center}
% DO NOT EDIT HERE. Instead edit macro in main.tex to keep metadata correct
  \papertitle 
\end{center}
}

\vspace*{2.0cm}

% Authors -------------------------------------------------
\begin{center}
%In the footnote, replace 'paper' by 'Letter' in case of submission to PRL or PLB 
% Edit macro in main.tex to keep metadata correct
\paperauthors\footnote{Authors are listed at the end of this paper.}
\end{center}

\vspace{\fill}

% Abstract -----------------------------------------------
\begin{abstract}
  \noindent
  The first untagged decay-time-integrated amplitude analysis of
  $\Bs \to \KS\Kpm\pimp$ decays is performed using a data sample
  corresponding to $3.0\invfb$ of $pp$ collision data recorded by the LHCb
  experiment during 2011 and 2012.
  The data are described by an amplitude model that contains contributions
  from the intermediate resonances $\Kstar(892)^{0,+}$, $K^*_2(1430)^{0,+}$ and
  $K^*_0(1430)^{0,+}$.
  Measurements of the decay modes $\Bs\to K^{*\pm}(892)\Kmp$ and
  $\Bs\to \KorKbar^{*0}(892)\KorKbar^{0}$ are in good agreement with, and
  more precise than, previous results. 
  The decays $\Bs\to \KstarIIpm\Kmp$ and
  $\Bs \to \KorKbar\!^{*}_{0}(1430)^0\KorKbar^{0}$ are observed for the
  first time with significance over 10 standard deviations.
%  a statistical significance of $17.3$ and $15.2\,\sigma$, respectively.  
  Systematic uncertainties associated to the contributions containing a tensor resonance preclude any statement on their significance.
\end{abstract}

\vspace*{2.0cm}

\begin{center}
  To be submitted to JHEP
\end{center}

\vspace{\fill}

{\footnotesize 
% Edit macro in main.tex to keep metadata correct
\centerline{\copyright~\papercopyright. \href{\paperlicenceurl}{\paperlicence}.}}
\vspace*{2mm}

\end{titlepage}


%%%%%%%%%%%%%%%%%%%%%%%%%%%%%%%%
%%%%%  EOD OF TITLE PAGE  %%%%%%
%%%%%%%%%%%%%%%%%%%%%%%%%%%%%%%%

%  empty page follows the title page ----
\newpage
\setcounter{page}{2}
\mbox{~}
%\newpage
%
%% Author List ----------------------------
%%  You need to get a new author list!
%\input{LHCb_authorlist.tex}
%
%The author list for journal publications is provided by the Membership Committee shortly after 'approval to go to paper' has been given.
%%It will be made available on the page
%%\verb!http://www.physik.uzh.ch/~strauman/forMemCo/LHCb-PAPER-XXXX-XXX/! .
%It will be sent to you by email shortly after a paper number has beens assigned.
%The author list should be included already at first circulation, 
%to allow new members of the collaboration to verify whether they have been included correctly.
%Occasionally a misspelled name is corrected or associated institutions become full members.
%In that case, a new author list will be sent to you.
%In case line numbering doesn't work well after including the authorlist, try moving the \verb!\bigskip! after the last author to a separate line.
%
%
%The authorship for Conference Reports should be ``The LHCb
%  collaboration'', with a footnote giving the name(s) of the contact
%  author(s), but without the full list of collaboration names.



\cleardoublepage









\renewcommand{\thefootnote}{\arabic{footnote}}
\setcounter{footnote}{0}

%%%%%%%%%%%%%%%%%%%%%%%%%
%%%%% Main text %%%%%%%%%
%%%%%%%%%%%%%%%%%%%%%%%%%

\pagestyle{plain} % restore page numbers for the main text
\setcounter{page}{1}
\pagenumbering{arabic}

%% Uncomment during review phase. 
%% Comment before a final submission.
\linenumbers

% You can include short sections directly in the main tex file.
% However, for larger papers it is desirable to split the text into
% several semiautonomous files, which can be revised independently.
% This is especially useful when developing a document in
% collaboration with several people, since then different parts can be
% edited independently.  This type of file organization is shown here.
% 

%
\section{Introduction}
\label{sec:Introduction}

The search for new sources of \CP violation in addition to that predicted by 
the CKM matrix~\cite{Cabibbo:1963yz,Kobayashi:1973fv} is among the main goals of current particle physics research.
One interesting approach is the study of decay-time distributions of neutral \B meson decays to hadronic final states mediated by the loop (``penguin'') $b \to s$ amplitude.  
As-yet undiscovered particles can contribute in the loops and cause the observables to deviate from their expected values 
in the Standard Model (SM)~\cite{Grossman:1996ke,Fleischer:1996bv,London:1997zk,Ciuchini:1997zp}.
Studies of various \Bz decays have been performed for this reason, including decay-time-dependent amplitude analyses of $B^0 \to \KS\pi^+\pi^-$~\cite{Dalseno:2008wwa,Aubert:2009me} and $B^0 \to \KS K^+K^-$~\cite{Nakahama:2010nj,Lees:2012kxa} transitions.
Such analyses, which involve describing the variation of the decay amplitudes over the full phase-space of the three-body decays, are more sensitive to interference effects than the ``quasi-two-body'' approach and hence are particularly important when broad resonances contribute.
Decay-time-dependent analyses of \Bs\ meson transitions mediated by hadronic $b \to s$ amplitudes have been performed for the $\Bs \to \Kp\Km$~\cite{LHCb-PAPER-2018-006}, $\Bs\to\phi\phi$~\cite{LHCb-PAPER-2014-026,LHCb-CONF-2018-001} and $\Bs\to\Kstarz\Kstarzb$~\cite{LHCb-PAPER-2017-048} decays, but not yet for any three-body \Bs\ decay.

The $\Bs\to\KS\Kpm\pimp$ channels have been observed~\cite{LHCb-PAPER-2013-042,LHCb-PAPER-2017-010}, and quasi-two-body measurements of the resonant contributions from $\Bs\to\Kstarpm\Kmp$~\cite{LHCb-PAPER-2014-043} and $\Bs\to\KstarzorKstarzb\KS$~\cite{LHCb-PAPER-2015-018} have also been performed.
These decays provide interesting long-term potential for time-dependent \CP\ violation measurements~\cite{Gronau:2006qn}.
The $\KS\Km\pip$ and $\KS\Kp\pim$ final states are both accessible to \Bs and \Bsb decays, with the corresponding amplitudes expected to be comparable in magnitude.
Example decay diagrams for contributions through the $\Bs \to \Kp\Kstarm~(\Kstarp\Km)$ and $\Bs \to \Kstarz\Kzb~(\Kz\Kstarzb)$ resonant processes are shown in Fig.~\ref{fig:feynman}, where subsequent transitions $\Kstarm \to \Kzb\pim$, $\Kstarz \to \Kp\pim$ and $\Kzb \to \KS$ and their conjugates lead to the $\KS\Kp\pim$ ($\KS\Km\pip$) final state for the former (latter) processes. 
The inclusion of charge conjugate processes is implied throughout the paper, except where explicitly stated otherwise.
Thus, large interference effects, and potentially large \CP\ violation effects, are anticipated.  
It is therefore of interest to study these channels with an amplitude analysis.

\begin{figure}[!tb]
  \centering
  \includegraphics*[width=0.46\textwidth]{figs/Generic_Bs2KstK_tree_ext}
  \includegraphics*[width=0.46\textwidth]{figs/Generic_Bs2KstK_loop}
  \includegraphics*[width=0.46\textwidth]{figs/Generic_Bs2KstK_loop_EW}
  \includegraphics*[width=0.46\textwidth]{figs/Generic_Bs2KsKz_loop}
  \caption{\small
    Feynman diagrams for (top left) external tree, (top right) internal penguin and 
    (bottom left) electroweak penguin contributions for $\Bs \to \Kp\Kstarm~(\Kstarp\Km)$ decays; and 
    (bottom right) internal penguin amplitude for the $\Bs \to \Kstarz\Kzb~(\Kz\Kstarzb)$ decay mode. 
    The electroweak penguin diagram for the $\Bs \to \Kstarz\Kzb~(\Kz\Kstarzb)$ channel is not shown; neither are diagrams corresponding to annihilation amplitudes. 
    In each case, the first set of final state particles (black) leads to the $\KS\Kp\pim$ final state, while the second set (blue) leads to $\KS\Km\pip$.
  }
  \label{fig:feynman}
\end{figure}
  
In this paper, the first Dalitz plot analysis of $\Bs\to\KS\Kpm\pimp$ decays is described.  
The analysis is based on a data sample corresponding to $3.0\invfb$ of $pp$ collision data recorded by the LHCb experiment during 2011 and 2012.
Due to the limited sample size available, the analysis is performed without considering decay-time dependence and without separating the $\Bs$ or $\Bsb$ initial states (\ie\ the analysis is ``untagged'').
Due to the modest effective tagging efficiency that can be achieved at hadron collider experiments, the inclusion of initial-state flavour tagging and decay-time dependence would not currently result in useful sensitivity to the additional parameters that can only be measured in such an analysis.
%Moreover, conservation of \CP\ violation in the decay amplitudes is assumed, in order to simplify the analysis by reducing the number of fitted parameters.

A novel feature of this analysis is that there are two independent final states ($\KS\Kp\pim$ and $\KS\Km\pip$) that are treated separately but simultaneously.
Denoting one by $f$ and the other by $\bar{f}$, then the former (latter) receives contributions from the amplitudes ${\cal{A}}_f$ and $\bar{\cal{A}}_f$ (${\cal{A}}_{\bar{f}}$ and $\bar{\cal{A}}_{\bar{f}}$), where ${\cal{A}}$ and $\bar{\cal{A}}$ are used to denote amplitudes for \Bs\ and \Bsb\ decays, respectively.
The untagged decay-time-integrated density of events in the Dalitz plot corresponding to $f$ therefore depends on $|{\cal A}_{f}|^2$ and $|\bar{{\cal A}}_{f}|^2$, while that for $\bar{f}$ depends on $|{\cal A}_{\bar{f}}|^2$ and $|\bar{{\cal A}}_{\bar{f}}|^2$.\footnote{
  The untagged decay-time-integrated rate also depends on an interference term that is responsible for the difference between the $t=0$ branching fraction and the decay-time-integrated branching fraction~\cite{DeBruyn:2012wj,LHCb-PAPER-2013-069,Dettori:2018bwt}.
  This must be considered when results are interpreted theoretically, but is not relevant for the discussion here.}
In the absence of \CP\ violation in decay ${\cal{A}}_f = {\bar{\cal{A}}}_{\bar{f}}$ and ${\bar{\cal{A}}}_f = {\cal{A}}_{\bar{f}}$, but there is no simple relation between ${\cal{A}}_f$ and ${\bar{\cal{A}}}_f$.
Indeed, theoretical predictions indicate that the values of these amplitudes could be quite different~\cite{Cheng:2014uga,Li:2014fla,Li:2018qrm}.
Thus, the situation differs from that usually considered in Dalitz plot analysis, where the density is given simply by the magnitude of a single amplitude squared. 

Precedent for handling this situation is taken from amplitude analyses of flavour-specific $B$ meson decays that do not account for \CP\ violation effects.  
In such analyses the distributions for $B$ and $\Bbar$ decays are added together, assuming them to be identical and fitting them with a single amplitude.
However, in the case that \CP\ violation effects are actually present, the true underlying density is given by the incoherent sum of two contributions, as here.  
In this example the fitted parameters of the amplitude model will differ from their true values by an amount that depends on the size of the \CP\ violation effects.
Similarly, by fitting each of the two $\Bs\to\KS\Kpm\pimp$ Dalitz plots with a single amplitude, the results will give values that differ from the true properties of the decays by amounts that must be estimated.
Detailed studies with simulated pseudo-experiments demonstrate that the fit fractions (defined in Sec.~\ref{sec:formalism}) obtained by this approach are biased by relatively small amounts that can be accounted for with systematic uncertainties, but that measurements of other quantities may not be reliable.
Therefore, the results of the analysis are presented in terms of fit fractions only.

The remainder of the paper is organised as follows.
In Sec.~\ref{sec:Detector}, a brief description of the LHCb detector, online selection algorithms and simulation software is given.
The selection of $\Bs\to\KS\Kpm\pimp$ candidates, and the method to estimate the signal and background yields are described in Sec.~\ref{sec:selection} and Sec.~\ref{sec:dataset}, respectively.
The analysis described in these chapters follows closely the methods used for the branching fraction measurement presented in Ref.~\cite{LHCb-PAPER-2017-010}.
As such, all four final states (\KsPiPi, \KsKpPim, \KsKmPip, and \KsKK, collectively referred to as $\Kshhp$ where $h$ represents either kaon or pion) are considered up to this point in the paper, but only $\KsKpPim$ and $\KsKmPip$ are discussed subsequently.
The Dalitz plot analysis formalism is given in Sec.~\ref{sec:formalism}, while inputs to the fit such as the signal efficiency and background distributions are contained in Sec.~\ref{sec:dalitz}.
Sources of systematic uncertainty are discussed in Sec.~\ref{sec:systematics} before the results are presented in Sec.~\ref{sec:results}.
A summary concludes the paper in Sec.~\ref{sec:summary}.


%
\section{Detector, trigger and simulation}
\label{sec:Detector}

The \lhcb detector~\cite{Alves:2008zz,LHCb-DP-2014-002} is a single-arm
forward spectrometer covering the \mbox{pseudorapidity} range $2<\eta <5$,
designed for the study of particles containing \bquark or \cquark quarks.
The detector includes a high-precision tracking system consisting of a
silicon-strip vertex detector (\velo) surrounding the $pp$ interaction
region, a large-area silicon-strip detector located upstream of a dipole
magnet with a bending power of about $4{\mathrm{\,Tm}}$, and three stations
of silicon-strip detectors and straw drift tubes placed downstream of the
magnet.
The tracking system provides a measurement of momentum, \ptot, of charged particles with
relative uncertainty that varies from 0.5\% at low momentum to 1.0\% at 200\gevc.
The minimum distance of a track to a primary vertex (PV), the impact parameter (IP), 
is measured with a resolution of $(15+29/\pt)\mum$,
where \pt is the component of the momentum transverse to the beam, in\,\gevc.
Different types of charged hadrons are distinguished using information
from two ring-imaging Cherenkov detectors.
Photons, electrons and hadrons are identified by a calorimeter system
consisting of scintillating-pad and preshower detectors, an electromagnetic
calorimeter and a hadronic calorimeter.
Muons are identified by a system composed of alternating layers of iron and
multiwire proportional chambers.

The online event selection is performed by a
trigger~\cite{LHCb-DP-2012-004}, 
which consists of a hardware stage, based on information from the calorimeter and muon
systems, followed by a software stage, in which all charged particles
with $\pt>500\,(300)\mevc$ are reconstructed for data collected in 2011\,(2012).
At the hardware trigger stage, events are required to contain a muon with high
\pt or a hadron, photon or electron with high transverse energy in the
calorimeters.
The software trigger requires a two-, three- or four-track secondary vertex
with significant displacement from all primary $pp$ interaction vertices.
At least one charged particle must have transverse momentum $\pt >
1.7\,(1.6)\gevc$ in the 2011\,(2012) data and be inconsistent with
originating from a PV.
A multivariate algorithm~\cite{BBDT} is used for the identification of
secondary vertices consistent with the decay of a \bquark hadron.
It is required that the software trigger decision must have been caused
entirely by tracks from the decay of the signal \B candidate.

Simulated data samples are used to investigate backgrounds from other
\bquark-hadron decays and also to study the detection and reconstruction
efficiency of the signal.
In the simulation, $pp$ collisions are generated using
\pythia~\cite{Sjostrand:2007gs,*Sjostrand:2006za} with a specific \lhcb
configuration~\cite{LHCb-PROC-2010-056}.
Decays of hadronic particles are described by \evtgen~\cite{Lange:2001uf},
in which final-state radiation is generated using
\photos~\cite{Golonka:2005pn}.
The interaction of the generated particles with the detector, and its
response, are implemented using the \geant toolkit~\cite{Allison:2006ve,
*Agostinelli:2002hh} as described in Ref.~\cite{LHCb-PROC-2011-006}.


%
\section{Event selection}
\label{sec:selection}

The selection requirements follow closely those used for the determination
of the branching fractions of the \BdstoKshhp decays, reported in
Ref.~\cite{LHCb-PAPER-2017-010}.
A brief summary of the requirements follows, with emphasis placed on
where they differ from those used in the branching fraction analysis.

Decays of \decay{\KS}{\pip\pim} are reconstructed in two different categories:
the first involving \KS mesons that decay early enough for the
resulting pions to be reconstructed in the \velo; and the
second containing those \KS mesons that decay later, such that track
segments from the pions cannot be formed in the \velo.
These categories are referred to as \emph{\LL} and \emph{\DD}, respectively.
While the \LL\ category has better mass, momentum and vertex resolution,
there are approximately twice as many \KS candidates reconstructed in the
\DD\ category.
In the following, \B candidates reconstructed from either a \LL\ or \DD\
\KS candidate, in addition to two oppositely charged tracks, are also
referred to with these category names.
In order to account for changes in the trigger efficiency for each of the
\KS reconstruction categories during the data taking, the data sample is
subdivided into 2011, 2012a, and 2012b data-taking periods.
The 2012b sample is the largest, corresponding to 1.4\invfb, and also has
the highest trigger efficiency.

To suppress backgrounds, in particular combinatorial background formed by
random combinations of unrelated tracks, the events satisfying the trigger
requirements are filtered by a loose preselection, followed by an optimised
multivariate selection.
The selection is designed in such a way to minimise correlation of the
signal efficiency with position in the Dalitz plot, resulting in better
control of the corresponding systematic uncertainties.
Consequenty, the selection exploits heavily the topological features
that arise from the detached vertex of the \B candidate and relies very
little on the kinematics of its decay products.

The preselection of \KS and \B candidates and the training of the
multivariate classifiers, based on a boosted decision tree~(BDT)
algorithm~\cite{Breiman,AdaBoost}, is identical to that reported in
Ref.~\cite{LHCb-PAPER-2017-010}.
The selection requirement placed on the output of the BDTs is independently
optimised for each data sample using a different figure of merit from that in Ref.~\cite{LHCb-PAPER-2017-010}.
A study was performed with pseudoexperiments that were generated using a model containing a set of resonances that might contribute to the \BstoKsKPi Dalitz plot, and signal and background yields corresponding to various possible selection requirements on the BDT output.
The statistical uncertainty on each of the magnitudes and phases of the
resonances in the model as well as the systematic uncertainty
corresponding to the knowledge of the Dalitz plot distribution of the
backgrounds were determined for each selection requirement.
The responses of several figures of merit were compared with the results of
this study and that which showed the closest correspondence to minimising
the uncertainties on the isobar parameters was chosen.
This figure of merit is
%
\begin{equation}
{\cal Q} \equiv \frac{N^2_{\rm sig}}{\left(N_{\rm sig}+N_{\rm bg}\right)^\frac{3}{2}} \,,
\end{equation}
%
where $N_{\rm sig}$ ($N_{\rm bg}$) represents the number of expected signal
(combinatorial background) events, for a given selection, in the signal region
defined as the invariant-mass window of five times the typical resolution
around the \Bd or \Bs mass.
The value of $N_{\rm sig}$ is estimated based on the known branching
fractions and efficiencies, while $N_{\rm bg}$ is calculated by fitting the
sideband above the signal region and extrapolating into the signal region.
It may be noted that ${\cal Q}$ is equal to the product of two other figures of merit considered in the literature: $N_{\rm sig}/\sqrt{N_{\rm sig}+N_{\rm bg}}$ (sometimes referred to as {\it significance}) and $N_{\rm sig}/\left(N_{\rm sig}+N_{\rm bg}\right)$ ({\it purity}).

Particle identification (PID) information is used to assign
each candidate exclusively to one out of four possible final states:
\KsPiPi, \KsKpPim, \KsKmPip, and \KsKK.
The PID requirements are optimised to reduce the cross-feed between the
different signal decay modes using the same figure of merit ${\cal Q}$ introduced for the BDT optimisation. 
Additional PID requirements are applied in order to reduce backgrounds from
decays such as \LbtoKsPip, where the proton is misidentified as a kaon.

Fully reconstructed \B-meson decays into two-body $\DorDsm\hadp$ or
$(\cquark\cquarkbar)\KS$ combinations, where $(\cquark\cquarkbar)$
indicates a charmonium resonance, may result in a \Kshhp final state that
satisfies the selection criteria and has the same \B-candidate
invariant mass distribution as the signal candidates.
The decays of \Lbbar\ baryons to $\Lcbar\hadp$ with \decay{\Lcbar}{\antiproton\KS} also
peak under the signal when the antiproton is misidentified.
A series of invariant mass vetoes, identical to those used in
Ref.~\cite{LHCb-PAPER-2017-010}, are employed to remove these backgrounds.

The fraction of selected events containing more than one \B candidate is
below the level of 1\%.
The candidate to be retained in each event is chosen randomly but in such a
way that the choice is reproducible.



%
\section{Determination of signal and background yields}
\label{sec:dataset}

The signal and background yields are determined by means of a simultaneous
unbinned extended maximum likelihood fit to the 24 \B-candidate invariant
mass distributions that result from considering separately the four final
states, three data-taking periods and two \KS reconstruction categories.
Three components contribute to each invariant mass distribution:
signal decays, backgrounds resulting from cross-feeds, and random
combinations of unrelated tracks.
The contribution from a fourth category of background, partially
reconstructed decays, is reduced to a negligible level by performing the
fit in the invariant mass window $5200 < m(\Kshhp) < 5800\mevcc$.
The modelling of each of the three fit components follows that used in
Ref.~\cite{LHCb-PAPER-2017-010}.
A brief summary of the models used is given here.

Signal decays with correctly identified final-state particles are
modelled with the sum of two Crystal Ball (CB)
functions~\cite{Skwarnicki:1986xj} that share common values for the mean
and width of the Gaussian part of the function but have independent power
law tails on opposite sides of the Gaussian peak.
Cross-feed contributions from misidentified $\BdstoKshhp$ decays are also modelled with the sum of two CB functions.
Only contributions with a single misidentified track are included since contributions from other potential misidentified decays are found to be negligibly small.
The yield of each misidentified decay is constrained, with respect to the
yield of the corresponding correctly identified decay, using the ratio of
the selection efficiencies and the corresponding uncertainty.
The combinatorial background is modelled by a linear function.

\begin{figure}[!tb]
\begin{center}
\includegraphics*[width=0.49\textwidth]{figs/fig2a}
\includegraphics*[width=0.49\textwidth]{figs/fig2b}
\end{center}
\caption{\small
  Invariant mass distribution of candidates in data for the
  (left)~\KsKpPim\ and (right)~\KsKmPip\ final states. 
  Components are detailed in the legend, where they are shown in the
  same order as they are stacked in the figure. 
}
\label{fig : mass-fit}
\end{figure}

The fit results for the \KsKpPim and \KsKmPip final states,
combining all data-taking periods and \KS reconstruction categories,
are shown in \fig{mass-fit}.
\tab{mass-fit} details the fitted yields of all categories, both in the
invariant mass region used for the mass fit and in the reduced region to be
used in the amplitude analysis, defined as $\mu\pm2.5\sigma$ where $\mu$
($\sigma$) is the fitted peak position (width) of the \Bs signal component in
that category.
The yields are given for each of the two final states broken down by
data-taking periods and \KS reconstruction categories.

%--------------------------------------------------------------------------------------------%
\begin{table}[t]
\caption{\small
  Yields obtained from the simultaneous fit to the invariant mass
  distribution of $\Bs\to\KS\Kpm\pimp$ candidates in data for each fit category: signal, combinatorial background and cross-feed from from misidentified $\BdstoKshhp$ decays.
  The uncertainties given on the yields in the full range are statistical only.
  Yields in the signal region $\pm2.5\sigma$ around the \Bs\ peak are also given; the determination of uncertainties on these values is described in Sec.~\ref{sec:systematics}.
}
\label{tab : mass-fit}
%\vspace{15pt}
\resizebox{\textwidth}{!}{
\begin{tabular}{ c | c | c | c  c | c  c | c c }
\hline
  Final			& \KS		&	& \multicolumn{2}{c}{\Bs\ signal}		& \multicolumn{2}{c}{Combinatorial}		& \multicolumn{2}{c}{Cross-feed}	\\
state			& category	& Year	& Full range			& $2.5\,\sigma$	& Full range			& $2.5\,\sigma$	& Full range		& $2.5\,\sigma$	\\
\hline                                                                                                                                                                                  
\mrsix{\KsKpPim}	& \mrthree{\DD}	& 2011	& $\pho73.6 \pm 10.6$		& $\pho72.1$	& $108.3 \pm 15.1$		& $22.1$	& $\pho8.9 \pm 2.8$	        & $1.7$		\\
			&		& 2012a	& $\pho48.2 \pm \pho8.6$	& $\pho45.7$	& $\pho70.1 \pm 12.1$		& $14.3$	& $\pho7.3 \pm 3.8$	        & $1.1$		\\
			&		& 2012b	& $135.3 \pm 13.6$		& $130.0$	& $\pho87.4 \pm 13.8$		& $17.9$	& $17.0 \pm 5.6$	& $3.1$		\\
\cdashline{2-9}                                                                                                                                                                         
			& \mrthree{\LL}	& 2011	& $\pho76.2 \pm \pho9.8$	& $\pho74.6$	& $\pho44.1 \pm \pho9.8$	& $\pho8.4$	& $\pho8.2 \pm 1.7$	        & $1.8$		\\
			&		& 2012a	& $\pho38.5 \pm \pho7.7$	& $\pho36.8$	& $\pho58.8 \pm 11.2$		& $11.2$	& $\pho7.8 \pm 1.8$	        & $0.9$		\\
			&		& 2012b	& $\pho73.5 \pm 10.6$		& $\pho71.9$	& $\pho71.7 \pm 13.1$		& $13.6$	& $15.9 \pm 2.5$	& $1.7$		\\
\cdashline{2-9}                                                                                                                                                                         
Total			&		&	&				& 431.1		&				& 87.5		&			& 10.3		\\
\hline                                                                                                                                                                                  
\mrsix{\KsKmPip}	& \mrthree{\DD}	& 2011	& $\pho72.8 \pm 10.3$		& $\pho71.4$	& $\pho78.9 \pm 12.7$		& $16.1$	& $\pho8.2 \pm 2.4$    	& $1.3$		\\
			&		& 2012a	& $\pho68.8 \pm \pho9.6$	& $\pho65.2$	& $\pho46.2 \pm \pho9.9$	& $\pho9.5$	& $\pho7.0 \pm 3.4$	        & $1.2$		\\
			&		& 2012b	& $165.1 \pm 15.2$		& $158.6$	& $104.1 \pm 15.0$		& $21.3$	& $17.3 \pm 5.8$	& $2.9$		\\
\cdashline{2-9}                                                                                                                                                                         
			& \mrthree{\LL}	& 2011	& $\pho77.3 \pm \pho9.8$	& $\pho75.7$	& $\pho39.0 \pm 10.2$		& $\pho7.4$	& $\pho9.6 \pm 1.7$	        & $1.4$		\\
			&		& 2012a	& $\pho40.3 \pm \pho8.1$	& $\pho38.5$	& $\pho58.9 \pm 11.9$		& $11.2$	& $\pho8.6 \pm 1.8$	        & $0.7$		\\
			&		& 2012b	& $\pho81.7 \pm 10.4$		& $\pho80.0$	& $\pho50.1 \pm 12.3$		& $\pho9.5$	& $15.0 \pm 2.5$	& $1.4$		\\
\cdashline{2-9}                                                                                                                                                                         
Total			&		&	&				& 489.4		&				& 75.0		&			& 8.9		\\
\hline
\end{tabular}
}
\end{table}
%--------------------------------------------------------------------------------------------%

%
\section{Dalitz plot analysis formalism}
\label{sec:formalism}

The Dalitz plot~\cite{Dalitz:1953cp} describes the phase-space of a three-body decay in terms of two of the three possible two-body invariant mass squared combinations.
In $\Bs \to \KS\Kpm\pimp$ decays the most significant resonant structures are expected to be from excited kaon states decaying to $\KS\pimp$ or $\Kpm\pimp$ and therefore these are used to define the Dalitz plot axes.
For a fixed $\Bs$ mass, these two invariant mass squared combinations can be used to calculate all other relevant kinematic quantities.

The Dalitz plot analysis involves developing a model that describes the variation of the complex decay amplitudes over the full phase-space of a three-body decay.
The distribution of decays seen in experiment is related to the square of the magnitude of the amplitude, modified to account for detection efficiency and background contributions.  
As described in Sec.~\ref{sec:Introduction}, this is only an approximation for $\Bs \to \KS\Kpm\pimp$ decays, where the physical distribution in each final state depends on the incoherent sum of two contributions.  
A single amplitude is nonetheless used to model the data, since it is not possible to separate the two contributing amplitudes without initial state flavour tagging; a systematic uncertainty is assigned to account for possible biases induced by this approximation.
The Dalitz plot fit is performed using the {\sc Laura++}~\cite{Laura++} package, with the different final states, \KS reconstruction categories and data-taking periods handled using the {\it J}{\sc fit} method~\cite{Ben-Haim:2014afa}.

The isobar model~\cite{Fleming:1964zz,Morgan:1968zza,Herndon:1973yn} is used to describe the complex decay amplitude. 
The total amplitude is given by the coherent sum of amplitudes from $N$ intermediate contributions,
% and is given by
\begin{equation}\label{eqn:amp}
  {\cal A}\left(m^2(\KS\pimp), m^2(\Kpm\pimp)\right) = \sum_{j=1}^{N} c_j F_j\left(m^2(\KS\pimp), m^2(\Kpm\pimp)\right) \,,
\end{equation}
where $c_j$ are complex coefficients describing the relative strength of each intermediate process. 
The resonant dynamics are contained in the $F_j\left(m^2(\KS\pimp),m^2(\Kpm\pimp)\right)$ terms, which are normalised such that the integral of the squared magnitude over the Dalitz plot is unity for each term.
For a $\KS\pimp$ resonance $F_j\left(m^2(\KS\pimp),m^2(\Kpm\pimp)\right)$ is given by
\begin{equation}
  \label{eq:ResDynEqn}
  F\left(m^2(\KS\pimp), m^2(\Kpm\pimp)\right) = 
  R\left(m(\KS\pimp)\right) \times X(|\vec{p}\,|\,r_{\rm BW}) \times X(|\vec{q}\,|\,r_{\rm BW}) 
  \times T(\vec{p},\vec{q}\,) \, ,
\end{equation}
where $\vec{p}$ is the bachelor particle\footnote{
  The ``bachelor'' particle is that not forming the resonance, \ie\ the $\Kpm$ in this example.} momentum and $\vec{q}$ is the momentum of one of the resonance daughters, both evaluated in the $\KS\pimp$ rest frame.
The $R$ functions are the mass lineshapes, typically described by the relativistic Breit--Wigner function with alternative shapes used in some specific situations.
The $X$ and $T$ terms describe barrier factors and angular distributions, respectively, and depend on the orbital angular momentum between the resonance and the bachelor particle, $L$;
the barrier factors $X$ are evaluated in terms of the Blatt--Weisskopf radius parameter $r_{\rm BW}$ for which a default value of $4.0 \gev^{-1}\hbar c$ is used.
The angular distributions are given in the Zemach tensor formalism~\cite{Zemach:1963bc,Zemach:1968zz}, and are proportional to the Legendre polynomials, $P_L(x)$, where $x$ is the cosine of the angle between $\vec{p}$ and $\vec{q}$ (referred to as the helicity angle). 
Detailed expressions for the functions $R$, $X$ and $T$ can be found in Ref.~\cite{Laura++}.

% The $X(z)$ terms are Blatt--Weisskopf barrier factors~\cite{blatt-weisskopf}, where $z=|\vec{q}\,|\,r_{\rm BW}$ or $|\vec{p}\,|\,r_{\rm BW}$ and $r_{\rm BW}$ is the barrier radius which is set to $4.0\gev^{-1}\approx 0.8\fm$~\cite{LHCb-PAPER-2014-036} for all resonances. 
% The barrier factors are angular momentum dependent and are given by
% \begin{equation}\begin{array}{rcl}
% L = 0 \ : \ X(z) & = & 1\,, \\
% L = 1 \ : \ X(z) & = & \sqrt{\frac{1 + z_0^2}{1 + z^2}}\,, \\
% L = 2 \ : \ X(z) & = & \sqrt{\frac{z_0^4 + 3z_0^2 + 9}{z^4 + 3z^2 + 9}}\,,\\
% L = 3 \ : \ X(z) & = & \sqrt{\frac{z_0^6 + 6z_0^4 + 45z_0^2 + 225}{z^6 + 6z^4 + 45z^2 + 225}}\,,
% \end{array}\label{eq:BWFormFactors}\end{equation}
% where $z_0$ is the value of $z$ at the pole mass of the resonance and $L$ is the orbital angular momentum between the resonance and the bachelor particle.
% Since the parent and daughter particles all have zero spin, $L$ is also the spin of the resonance.

% The $T(\vec{p},\vec{q})$ terms describe the angular distributions in the Zemach tensor formalism~\cite{Zemach:1963bc,Zemach:1968zz} and are given by
% \begin{equation}\begin{array}{rcl}
% L = 0 \ : \ T(\vec{p},\vec{q}) & = & 1\,,\\
% L = 1 \ : \ T(\vec{p},\vec{q}) & = & -\,2\,\vec{p}\cdot\vec{q}\,,\\
% L = 2 \ : \ T(\vec{p},\vec{q}) & = & \frac{4}{3} \left[3(\vec{p}\cdot\vec{q}\,)^2 - (|\vec{p}\,||\vec{q}\,|)^2\right]\,,\\
% L = 3 \ : \ T(\vec{p},\vec{q}) & = & -\,\frac{24}{15} \left[5(\vec{p}\cdot\vec{q}\,)^3 - 3(\vec{p}\cdot\vec{q}\,)(|\vec{p}\,||\vec{q}\,|)^2\right]\,.
% \end{array}\label{eq:ZTFactors}\end{equation}
% These expressions are proportional to the Legendre polynomials, $P_L(x)$, where $x$ is the cosine of the angle between $\vec{p}$ and $\vec{q}$ (referred to as the helicity angle). 

% The $R\left(m(\KS\pimp)\right)$ functions are the mass lineshapes.
% Resonant contributions are typically described by the relativistic Breit--Wigner (RBW) function
% \begin{equation}
% \label{eq:RelBWEqn}
% R(m) = \frac{1}{(m_0^2 - m^2) - i\, m_0 \Gamma(m)} \,,
% \end{equation}
% where the mass-dependent decay width is
% \begin{equation}
% \label{eq:GammaEqn}
% \Gamma(m) = \Gamma_0 \left(\frac{q}{q_0}\right)^{2L+1}
% \left(\frac{m_0}{m}\right) X^2(q\,r_{\rm BW}) \,,
% \end{equation}
% where $q_0$ is the value of $q = |\vec{q}\,|$ when the invariant mass is equal to the pole mass of the resonance, $m_0$.

% The large phase space available in \B decays allows for the presence of nonresonant amplitudes (\ie\ contributions that do not proceed via a known resonance) that vary across the Dalitz plot.
% An exponential form factor (EFF) has been found to describe nonresonant contributions well in several Dalitz plot analyses of \B decays~\cite{Garmash:2004wa}, 
% \begin{equation}
%   R(m) = \exp\left[-\alpha m^2\right] \, ,
%   \label{eq:nonres}
% \end{equation}
% where $\alpha$ is a shape parameter that must be determined from the data and $m$ is a two-body invariant mass ($m(\KS\pimp)$ in this example).

% The RBW function is a good model for narrow resonances that are well separated from any other resonant or nonresonant
% contribution of the same spin.
% This approach is known to break down in the $K\pi$ S-wave because the $\KstarIIb$ resonance interferes strongly with a slowly varying nonresonant term, as described in Ref.~\cite{Meadows:2007jm}.
% The LASS lineshape~\cite{lass} has been developed to combine these two contributions,
% \begin{eqnarray}
%  \label{eq:LASSEqn}
%   R(m) & = & \frac{m}{q \cot{\delta_B} - iq} + \exp\left[2i \delta_B\right]
%   \frac{m_0 \Gamma_0 \frac{m_0}{q_0}}
%        {(m_0^2 - m^2) - i m_0 \Gamma_0 \frac{q}{m} \frac{m_0}{q_0}}\, , \\
% {\rm where} \ \cot{\delta_B} & = & \frac{1}{aq} + \frac{1}{2} r q \, ,
% \end{eqnarray}
% and where $m_0$ and $\Gamma_0$ are the pole mass and width of the $\KstarIIb$ state, and $a$ and $r$ are shape parameters.

The complex coefficients $c_j$, defined in Eq.~(\ref{eqn:amp}), are determined from the fit to data.
These are used to obtain fit fractions for each component $j$, which provide a robust and convention-independent way to report the results of the analysis.
The fit fractions are defined as the integral over one Dalitz plot ($\KS\Kp\pim$ or $\KS\Km\pip$) of the amplitude for each intermediate component squared, divided by that of the coherent matrix element squared for all intermediate contributions, 
\begin{equation}
{\it FF}_j =
\frac
{\int\!\!\int_{\rm DP}\left|c_j F_j\right|^2~dm^2(\KS\pimp)\,dm^2(\Kpm\pimp)}
{\int\!\!\int_{\rm DP}\left|{\cal A}\right|^2~dm^2(\KS\pimp)\,dm^2(\Kpm\pimp)} \, ,
\label{eq:fitfraction}
\end{equation}
where the dependence of $F_j$ and ${\cal A}$ on Dalitz plot position has been omitted for brevity.
The fit fractions need not sum to unity due to possible net constructive or destructive interference.
% , described by interference fit fractions defined by 
% \begin{equation}
%   {\it FF}_{ij} =
%   \frac
%   {\int\!\!\int_{\rm DP} 2 \, \Real\left[c_ic_j^* F_iF_j^*\right]~dm^2(\KS\pimp)\,dm^2(\Kpm\pimp)}
%   {\int\!\!\int_{\rm DP}\left|{\cal A}\right|^2~dm^2(\KS\pimp)\,dm^2(\Kpm\pimp)} \, ,
%   \label{eq:intfitfraction}
% \end{equation}
% where the dependence of $F_i^{(*)}$ and ${\cal A}$ on the Dalitz plot position has been omitted.
% By definition, the sum of all fit fractions and interference fit fractions
% with $i<j$ only is equal to unity.  

%% For the purpose of the branching ratio calculation, it is useful to consider quantities averaged over the two different final states.
For this analysis, it is useful to define also flavour-averaged fit fractions $\widehat{FF}_{j}$, where the numerator and denominator of Eq.~(\ref{eq:fitfraction}) are replaced by sums of the same quantities over both final states, and it is understood that a resonance corresponding to $j$ in one Dalitz plot will be replaced by its conjugate in the other (\eg\ \KstarIm\ in the $\KS\Kp\pim$ final state and \KstarIp\ for $\KS\Km\pip$).
These can be converted into product of branching fractions for the \Bs\ and \Kstar\ decays by multiplying by the known $\BsToKzBarOptKpi$ branching fraction,
\begin{equation}
  \label{eq:Chap-FFhat}
  \Br{\Bs \to \Kstar\kaon; \Kstar\to\kaon\pion} = \widehat{FF}_{j} \times \Br{\BsToKzBarOptKpi} \,, 
\end{equation}  
where $\Br{\Bs \to \Kstar\kaon}$ is the sum of the branching fractions for the two conjugate final states.

%
\section{Dalitz plot fit}
\label{sec:dalitz}

The parameters of the signal model are determined from an unbinned maximum likelihood fit to the distributions of data across the $\KS\Kp\pim$ and $\KS\Km\pip$ Dalitz plots.  
The physical signal model is modified to account for variation of the efficiency across the phase space, and background contributions are accounted for.
The yields of signal and background components in the signal region are taken from Table~\ref{tab : mass-fit}.
Separate efficiency functions and background models for each final state, \KS reconstruction category and data-taking period are also used.

Since in general the resonance masses are much smaller than the \Bs\ mass, 
the selected candidates tend to populate regions close to the kinematic boundaries of the Dalitz plot.
Therefore, it is convenient to describe the signal efficiency variation and background event density using the transformed coordinates referred to as square Dalitz plot (SDP) variables, defined by
%%
\begin{equation}
\label{eq:sqdp-vars}
m^{\prime} \equiv \frac{1}{\pi}
\arccos\left(2\frac{m(\Kpm\pimp) - m^{\rm min}_{\Kpm\pimp}}{m^{\rm max}_{\Kpm\pimp} - m^{\rm min}_{\Kpm\pimp}} - 1 \right)\,, 
\qquad
\theta^{\prime} \equiv \frac{1}{\pi}\theta(\Kpm\pimp)\,,
\end{equation}
%%
where $m(\Kpm\pimp)$ is the invariant mass of the charged kaon and pion,
$m^{\rm max}_{\Kpm\pimp} = m_{\Bs} - m_{\KS}$ and
$m^{\rm min}_{\Kpm\pimp} = m_{\Kpm} + m_{\pimp}$
are the kinematic limits of $m_{\Kpm\pimp}$,
and $\theta(\Kpm\pimp)$ is the helicity angle between the \pimp\ and the \KS\ in the $\Kpm\pimp$ rest frame.
% These variables have validity ranges between 0 and 1.

\subsection{Signal efficiency variation}

The signal efficiency is determined accounting for effects due to the LHCb detector geometry, and due to reconstruction and selection requirements.
The effects of PID requirements are considered separately to the rest of the selection efficiency to facilitate the use of data-driven methods. 

The geometric efficiency is determined from generator-level simulation.
This contribution is the same for the 2012a and 2012b samples, and for the \LL and \DD categories, as it is purely related to the kinematics of the \Bs\ mesons produced in LHC collisions.
The effect is, however, evaluated separately for 2011 and 2012 data due to the different beam energy.

The reconstruction and selection (excluding PID) efficiency is determined from simulated samples, now also accounting for the response of the detector.
Small corrections due to known differences between data and simulation in the track finding efficiency~\cite{DeCian:1402577} and hardware trigger response~\cite{MartinSanchez:1407893} are applied.

The efficiency of the PID requirements is determined from large control samples of $\Dstarp \to \Dz \pip$, $\Dz \to \Km\pip$ decays.
Differences in kinematics and detector occupancy between the control samples and the signal data are accounted for~\cite{LHCb-DP-2012-003,LHCb-PUB-2016-021}.

The combined efficiency maps are obtained as products of SDP histograms describing each of the three contributions described above.
These are subsequently smoothed using two-dimensional cubic splines.
The variation of the efficiency across the SDP is similar for each subsample of the data; the absolute scale differs between \LL and \DD categories due to acceptance and between data-taking periods due to changes in the trigger.
The efficiency is lowest for large values of $m^{\prime}$, with a peak at $m^{\prime} \sim 0.3$; there is about a factor of five difference in the efficiency between these two regions mainly caused by the difficulty to reconstruct decays in a region of phase space where the $\Kpm$ and $\pimp$ tracks are soft and the $\KS$ is energetic.

\subsection{Background modelling}

As can be seen in Fig.~\ref{fig : mass-fit} and Table~\ref{tab : mass-fit}, the signal region contains contributions from combinatorial background and cross-feed from misidentified $\Bz\to\KS\pip\pim$ decays.
The Dalitz plot distribution of the combinatorial background is modelled using data from a sideband at high $m(\KS\Kpm\pimp)$. %$5400 < m(\KS\Kpm\pimp) < 5800 \mevcc$. 
In order to increase the size of the sample used for this modelling, a looser BDT requirement is imposed than that for the signal selection.
It is verified that this does not change the Dalitz plot distribution of the background significantly, as it should not since the BDT is explicitly constructed to minimise correlation of its output variable with position in the Dalitz plot.
The combinatorial background is found to vary smoothly over the Dalitz plot.

Cross-feed from misidentified $\Bz\to\KS\pip\pim$ decays is modelled using a simulation of this decay, weighted in order to reproduce its measured Dalitz plot distribution~\cite{Aubert:2009me}.
The effect of the detector response is simulated, with the effect of the PID requirements accounted for by weights determined from data control samples, similarly as done for the evaluation for the signal efficiency.  
The most prominent structures in the Dalitz plot model for this background are due to the \KstarIpm\ resonances.

\subsection{Amplitude model for $\Bs \to \KS\Kpm\pimp$ decays} 

The Dalitz plot distributions of the selected $\Bs\to\KS\Kpm\pimp$ candidates, after background-subtraction and efficiency-correction and for all data subsamples combined, are shown in Fig.~\ref{fig:dp-distribution}.
There are clear excesses at low values of both $m^2(\KS\pimp)$ and $m^2(\Kpm\pimp)$, corresponding to excited kaon resonances.
There is no strong excess at low values of $m^2(\KS\Kpm)$, which would appear as diagonal bands towards the upper right of the kinematically allowed regions of the Dalitz plots.
The two Dalitz plot distributions appear to be consistent with each other, and hence with \CP\ conservation.

\begin{figure}[!tb]
  \begin{center}
    \includegraphics*[width=0.49\textwidth]{figs/DP_Bs2KSKpi}
    \includegraphics*[width=0.49\textwidth]{figs/DP_Bs2KSpiK}
  \end{center}
\caption{\small
  Background-subtracted and efficiency-corrected Dalitz plot distributions for (left) \KsKpPim and (right) \KsKmPip final states. 
  Boxes with a cross indicate negative values.
  }
  \label{fig:dp-distribution}
\end{figure}

The baseline signal model is developed by considering the impact of including or removing resonant or nonresonant contributions in the model.
The kaon resonances listed in Ref.~\cite{PDG2017} are considered, with charged and neutral isospin partners treated separately, as it is possible that one contributes significantly while the other does not.
If a resonance is included in the model for one final state, its conjugate is always also included in the model for the other, however.  
States which can decay to $\KS\Kpm$, such as the $a_2(1320)^\pm$ particle, are also considered but none are found to contribute significantly.

The baseline model contains contributions from the $\Kstar(892)^{0,+}$, $K^*_0(1430)^{0,+}$ and $K^*_2(1430)^{0,+}$ resonances.
The vector and tensor states are described with relativistic Breit--Wigner functions with parameters taken from Ref.~\cite{PDG2017}.
This is not appropriate for the broad $K\pi$ S-wave.
Several different lineshapes that have been suggested in the literature are tested, with the LASS description~\cite{lass} found to be most suitable in terms of fit stability and agreement with the data.  
This combines the $K^*_0(1430)$ resonance with a slowly varying nonresonant component; the associated parameters are taken from Refs.~\cite{PDG2017,lass2}.

The $\Bs\to \KstarIpm\Kmp$ and $\Bs \to \KstarIzoptbar\KorKbarz$ decays have previously been observed~\cite{LHCb-PAPER-2014-043,LHCb-PAPER-2015-018}.\footnote{
  The notation $\KstarIzoptbar\KorKbarz$ refers to the sum of the $\KstarIz\Kzb$ and $\KstarIzb\Kz$ final states, \etc
}
The significance of each of the other contributions is evaluated using a likelihood ratio test.  
Ensembles of simulated pseudoexperiments are generated with parameters corresponding to the best fit to data obtained with models that do not contain the resonance of interest, but that otherwise contain the same resonances as the baseline model.
Each pseudoexperiment is fitted with models both with and without the given resonance included, from which a distribution of the difference in negative log likelihood is obtained.
This is found to be well fitted by a $\chisq$ shape, which can then be extrapolated to find the $p$-value corresponding to the difference in negative log likelihood obtained in data.  

Using this procedure, the significances for the \KstarIIp, \KstarIIz, \KstarIIIp\ and \KstarIIIz\ contributions are found to correspond to 17.3, 15.2, 4.0 and 4.8 standard deviations, when only statistical uncertainties are included.
Among all the systematic variations discussed in Sec.~\ref{sec:systematics}, the $K\pi$ S-wave contributions remain highly significant, and therefore the $\Bs\to \KstarIIpm\Kmp$ and $\Bs \to \KstarIIzoptbar\KorKbarz$ decays are considered to be observed with significance over 10 standard deviations.
Some systematic variations do, however, impact strongly on the need to include tensor resonances in the fit model, and thus preclude any similar conclusion for the $\Bs\to \KstarIIIpm\Kmp$ and $\Bs \to \KstarIIIzoptbar\KorKbarz$ decays.

\begin{figure}[!tb]
  \begin{center}
    \includegraphics*[width=0.47\textwidth]{figs/m12_Bs2KSKpi_Combined.pdf}
    \includegraphics*[width=0.47\textwidth]{figs/m12_Bs2KSpiK_Combined.pdf}
    \includegraphics*[width=0.47\textwidth]{figs/m23_Bs2KSKpi_Combined.pdf}
    \includegraphics*[width=0.47\textwidth]{figs/m13_Bs2KSpiK_Combined.pdf}
    \includegraphics*[width=0.47\textwidth]{figs/m13_Bs2KSKpi_Combined.pdf}
    \includegraphics*[width=0.47\textwidth]{figs/m23_Bs2KSpiK_Combined.pdf}
  \end{center}
\caption{\small
  Invariant mass data distribution for (top)~$m(\Kpm\pimp)$, (middle)~$m(\KS\pimp)$ and (bottom)~$m(\KS\Kpm)$. 
  The data are shown with black points, while the full fit is shown in blue, background from combinatorial background in red and \BdToKSpipi\ cross-feed in green.
  The resonance components are shown with
  $\KstarIpm$ in violet dash triple-dotted, $\KstarIIpm$ in orange dotted, $\KstarIIIpm$ in magenta long-dashed, 
  $\KstarIzoptbar$ in dark cyan dash dotted, $\KstarIIzoptbar$ in green long-dash dotted and $\KstarIIIzoptbar$ gray long-dash double-dotted lines. 
  }
  \label{fig:dp-fits}
\end{figure}

The results of the fit of the baseline model to the data are shown in Fig.~\ref{fig:dp-fits}.
Various methods are used to assess the goodness-of-fit~\cite{Williams:2010vh} and find good agreement between the model and the data.
The results for the fit fractions are given in Table~\ref{tab:FFs}.
The statistical uncertainties on the fit fractions are evaluated from the spreads in these values obtained when fitting ensembles of pseudoexperiments generated according to the baseline model with parameters corresponding to those obtained in the fit to data.
The fit fractions for each resonance and its conjugate (in the other Dalitz plot) are consistent, as expected from the absence of difference between the two Dalitz plot distributions.
Thus, no significant \CP\ violation effect is observed.

\begin{table}[!tb]
\centering
\caption{\small
  Results of the fit with the baseline model to the $\KS\Kp\pim$ and $\KS\Km\pip$ Dalitz plots.
  The fit fractions associated with each resonant component are given with statistical uncertainties only.
  The sums of fit fractions for both $\Bs \to \KS\Kp\pim$ and $\Bs \to \KS\Km\pip$ are 102\%, corresponding to low net interference effects.
  }
\label{tab:FFs}
\begin{tabular}{lclc}
\hline \\ [-2.4ex]   
\multicolumn{2}{c}{$\Bs \to \KS\Kp\pim$} & \multicolumn{2}{c}{$\Bs \to \KS\Km\pip$} \\
Resonance   & Fit fraction (\%)          & Resonance   & Fit fraction (\%)          \\
\hline \\ [-2.4ex]                                 
\KstarIm    & $15.6 \pm 1.5$             & \KstarIp    & $13.4 \pm 2.0$             \\
\KstarIIm   & $30.2 \pm 2.6$             & \KstarIIp   & $28.5 \pm 3.6$             \\
\KstarIIIm  & $\,\,\,2.9 \pm 1.3$        & \KstarIIIp  & $\,\,\,5.8 \pm 1.9$        \\
\KstarIz    & $13.2 \pm 2.4$             & \KstarIzb     & $19.2 \pm 2.3$             \\                                      
\KstarIIz   & $33.9 \pm 2.9$             & \KstarIIzb    & $27.0 \pm 4.1$             \\
\KstarIIIz  & $\,\,\,5.9 \pm 4.0$        & \KstarIIIzb   & $\,\,\,7.7 \pm 2.8$        \\
\hline
\end{tabular}
\end{table}

%
\section{Systematic uncertainties}
\label{sec:systematics}

Systematic uncertainties that affect the determination of the observables in the amplitude analysis arise from inaccuracy in the experimental inputs and the choice of the baseline amplitude parametrisation. 
The evaluation of effects arising from these sources is discussed in the following, with a summary of the systematic uncertainties on the fit fractions in Table~\ref{tab:sys-summary}. 
%Note that the information on the uncertainties on the magnitudes and phases of the complex amplitude have been omitted, since only branching fractions are reported as results. 
%Experimental systematic uncertainties originate from the imprecision introduced from fixed signal and background yields, efficiency and background phase-space modelling, 
%and possible bias due to the fitting procedure. 
%Model uncertainties account for fixed parameters in the signal model, the addition or removal of signal components in the fit, choice of signal model, 
%and the approximation of the flavoured averaged amplitude fit.

%
%
\begin{table}[tb]
\centering
\caption{\small
  Systematic uncertainties on the fit fractions, quoted as absolute uncertainties in \%. 
  The columns give the contributions from each of the different sources described in the text.
}
\label{tab:sys-summary}
\vspace{12pt}
\resizebox{\textwidth}{!}{
\begin{tabular}{lcccccccc|c}
\hline
& \multicolumn{9}{c}{Fit fraction (\%) uncertainties} \\
  Resonance   & Yields & Bkg. & Eff. & Fit bias & Add.\ res. & Fixed par. & Alt.\ model &  Method & Total  \\
\hline                                                                     
\KstarIm    & $0.2$   &   $0.2$   &   $0.5$  &   $0.2$   &   --      &   $0.7$    &   $5.4$   &   $3.1$   &   $6.3$ \\  
\KstarIIm   & $0.1$   &   $0.2$   &   $0.6$  &   $0.3$   &   $0.1$   &   $2.1$    &   $22.0$  &   $2.9$   &   $22.3$ \\
\KstarIIIm  & $0.1$   &   $0.1$   &   $0.3$  &   $0.6$   &   $0.1$   &   $1.8$    &   $2.2$   &   $0.2$   &   $2.9$ \\
\KstarIz    & $0.2$   &   $0.2$   &   $0.4$  &   $0.9$   &   --      &   $0.3$    &   $7.0$   &   $2.0$   &   $7.4$ \\
\KstarIIz   & $0.2$   &   $0.3$   &   $0.9$  &   $0.4$   &   $0.1$   &   $4.4$    &   $3.3$   &   $1.3$   &   $5.7$ \\
\KstarIIIz  & $0.1$   &   $0.3$   &   $0.7$  &   $1.3$   &   $0.2$   &   $4.4$    &   $3.6$   &   $1.0$   &   $6.0$ \\
\hline                                                                                                                
\KstarIp    & $0.4$   &   $0.1$   &   $0.6$  &   $0.5$   &   $0.1$   &   $0.7$    &   $1.1$   &   $0.7$   &   $1.8$ \\
\KstarIIp   & $0.5$   &   $0.4$   &   $0.7$  &   $0.8$   &   $0.2$   &   $6.4$    &   $13.0$  &   $4.5$   &   $15.2$ \\
\KstarIIIp  & $0.1$   &   $0.2$   &   $0.4$  &   $0.2$   &   $0.1$   &   $4.1$    &   $4.5$   &   $3.2$   &   $6.9$ \\
\KstarIzb   & $0.4$   &   $0.3$   &   $0.4$  &   $0.2$   &   $0.2$   &   $0.5$    &   $3.0$   &   $7.9$   &   $8.5$ \\
\KstarIIzb  & $0.4$   &   $0.4$   &   $0.6$  &   $0.8$   &   $0.7$   &   $0.9$    &   $3.9$   &   $5.4$   &   $6.8$ \\
\KstarIIIzb & $0.1$   &   $0.2$   &   $0.4$  &   $0.8$   &   $0.1$   &   $1.0$    &   $5.5$   &   $2.7$   &   $6.3$ \\
\hline        
\end{tabular}
}
\end{table}
%-------------------------------------------------------------------------------

Uncertainties associated to the signal and background yields obtained from the mass fit are examined from scaling the errors obtained from the whole mass fit range to the signal region. 
Statistical uncertainties on the yields are obtained from the covariance matrix of the baseline fit result, and systematic uncertainties are extracted similarly as for the branching fraction measurement~\cite{LHCb-PAPER-2013-042}.  
A series of pseudo-experiments are generated from the baseline mass fit which are fitted by varying all of the fixed parameters according to their covariance matrix. 
The differences in results between the toy and baseline fit ensembles are then fitted with a Gaussian function, and  
a systematic uncertainty is assigned as the linear sum of the absolute value of the corresponding mean and width. 
The dependence on the models used in the invariant mass fit is investigated by repeating the fit on ensembles generated with alternative shapes. 
The signal shape is examined by removing the right-tail of the mass distribution whilst 
for the combinatorial background the effect of floating independently the slopes for each spectrum and replacing the exponential by a linear model are evaluated.  
These uncertainties are propagated into the amplitude fit by generating a series of ensembles in order to address the uncertainties related to the yield extraction, either by the RMS of the fitted quantity over the ensemble or the mean difference to the baseline model. 

Uncertainties arising from the modelling of the Dalitz plot distributions of both combinatorial and cross-feed backgrounds are estimated 
by varying the histograms used to describe these shapes within their statistical uncertainties to create an ensemble of new histograms.  
The data is refitted using each new histogram and the systematic uncertainty is taken from the RMS of the fitted quantity over the ensemble.

Effects related to the efficiency modelling are determined by repeating the Dalitz plot fit using new histograms obtained in a similar fashion as for the background. 
Uncertainties caused by residual disagreements between data and simulation are addressed by examining 
alternative efficiency maps either by varying the binning scheme choice or using different corrections.
The simulated distributions of the features used in the BDT algorithm are known to have residual differences with respect to the signal data. 
The impact of this is estimated by repeating the multivariate classifier training using 
signal samples corrected by a multivariate reweighting procedure~\cite{Rogozhnikov:2016bdp}.
%from uncorrelated background-subtracted samples~\cite{Pivk:2004ty}.  
Potential disagreements in the vertexing of the \KS meson as a function of momentum are also 
studied using $D^{*+}\to (D^{0}\to \phi \KS)\pi^{+}$ calibration samples, with a similar procedure 
as in Ref.~\cite{LHCb-PAPER-2012-009}.
Finally, effects related to the hardware stage trigger are addressed
by calibrating the associated efficiency maps using $B \to \jpsi K$ and $\jpsi \kaon\pi$ control samples.  
The data fit is repeated including each of these new efficiency models and a systematic uncertainty is assigned from the mean difference to the results with the baseline model. 

Pseudoexperiments generated from the baseline fit results are used to quantify any intrinsic bias in the fit procedure. 
The uncertainties are evaluated as the sum in quadrature of the mean difference between the baseline and sampled values and the corresponding uncertainty. 

The choice of the baseline Dalitz fit model introduces important uncertainties through the choices of both the resonant or nonresonant contributions included and the lineshapes used.
The effects on the results of including additional $K^{*}(1410)$, $K^{*}(1680)$ or $a_{2}(1320)^{\pm}$ signal components in the fit are examined individually for each contribution. 
Some alternative fits give unrealistic results (for example, with very large sums of fit fractions) and are not included in the evaluation of this uncertainty.

Each resonant contribution has fixed parameters in the fit, which are varied to evaluate the associated systematic uncertainties. 
These include masses and widths~\cite{PDG2017} and the effective range and scattering length parameters of the LASS lineshape~\cite{PDG2017,lass2}.
The Blatt--Weisskopf radius parameter is varied within the range $3.0$--$5.0 \gev^{-1}\hbar c$.
The fit is repeated many times varying each of these fixed parameters within its uncertainties. 
The RMS of the distribution of the change in each fitted parameter is taken as the systematic uncertainty.

The baseline LASS parametrisation for the $K\pi$ S-wave modelling is known to be an approximate form, 
and associated uncertainties are assigned by evaluating the impact of an alternative parametrisation. 
This component is replaced by the model suggested in Ref.~\cite{ElBennich:2009da}, using tabulated magnitudes and phases at various values of $m(K\pi)$ obtained from form factors. 
This is found to provide a good description of the data, despite the larger interference pattern observed. 
Further theoretical work is required to have an accurate description of the S-wave term, therefore 
the differences between this alternative model and the baseline model are conservatively assigned as systematic uncertainties.

Modelling each of the $\Bs\to\KS\Kpm\pimp$ Dalitz plots with a single amplitude is an approximation, as discussed in Sec.~\ref{sec:Introduction}.
The associated systematic uncertainty is evaluated by generating with a full decay-time-dependent model a series of ensembles with different parameters for the contributing amplitudes based on the expected branching fractions~\cite{Cheng:2014uga,Li:2014fla} and a range of different \CP\ violation hypotheses.
The results obtained from the fit with the approximate model are compared to those expected with the full model, with results for the fit fractions found to be robust in contrast to those for relative phases between resonant contributions.
The systematic uncertainty is assigned as the bias found in the case that the model is generated with the theoretically preferred values for the parameters~\cite{Cheng:2014uga,Li:2014fla}.
% different \text{CP}-violating hypotheses, \textit{e.g} ${\cal{A}}^{CP} = 0, \pm 10, \pm 20\,\%$ in $B^{0}_{s}\to K^{+}K^{*-}$ and $B^{0}_{s}\to K^{*+}K^{-}$~\cite{Cheng:2014uga,Li:2014fla} decays. 
% There is a good agreement in the flavoured averaged fit fractions obtained in all cases, 
% and thus a conservative systematic uncertainty using the most theoretical motivated scenario is added to the measurement. 
%
%-------------------------------------------------------------------------------


%
\section{Results}
\label{sec:results}

The flavour-averaged fit fractions are converted into product branching fractions using Eq.~(\ref{eq:Chap-FFhat}) and 
${\cal B}(\BsToKzBarOptKpi) = (84.3 \pm 3.5 \pm 7.4 \pm 3.4)\times10^{-6}$~\cite{LHCb-PAPER-2017-010}, to obtain
\begin{equation*}
\begin{array}{rcl}      
\Br{\Bs \to \KstarIpm\Kmp; \KstarIpm \to \KorKbarz\pipm}               &=& \\
& \multicolumn{2}{l}{\hspace{-4mm}(   12.4 \pm   0.8   \pm   0.5   \pm   \phz2.7   \pm   1.3)  \times 10^{-6} \,,} \\  
\Br{\Bs \to \KpiSpm\Kmp}                                               &=& \\
& \multicolumn{2}{l}{\hspace{-4mm}(   24.9 \pm   1.8   \pm   0.5   \pm      20.0   \pm   2.6)  \times 10^{-6} \,,} \\
\Br{\Bs \to \KstarIIIpm\Kmp; \KstarIIIpm \to \KorKbarz\pipm}           &=& \\
& \multicolumn{2}{l}{\hspace{-4mm}(\phz3.4 \pm   0.8   \pm   0.4   \pm   \phz5.4   \pm   0.4)  \times 10^{-6} \,,} \\
\Br{\Bs \to \KstarIzoptbar\KorKbarz; \KstarIzoptbar \to \Kmp\pipm}     &=& \\
& \multicolumn{2}{l}{\hspace{-4mm}(   13.2 \pm   1.9   \pm   0.8   \pm   \phz2.9   \pm   1.4)  \times 10^{-6} \,,} \\
\Br{\Bs \to \KpiSz\KorKbarz}                                           &=& \\
& \multicolumn{2}{l}{\hspace{-4mm}(   26.2 \pm   2.0   \pm   0.7   \pm   \phz7.3   \pm   2.8)  \times 10^{-6} \,,} \\
\Br{\Bs \to \KstarIIIzoptbar\KorKbarz; \KstarIIIzoptbar \to \Kmp\pipm} &=& \\
& \multicolumn{2}{l}{\hspace{-4mm}(\phz5.6 \pm   1.5   \pm   0.6   \pm   \phz7.0   \pm   0.6)  \times 10^{-6} \,,}
\end{array}  
\end{equation*}
%-------------------------------------------------------------------------------
where the uncertainties are respectively statistical, systematic related to experimental and model uncertainties, and due to
the uncertainty on ${\cal{B}}(\BsToKzBarOptKpi)$.\footnote{The notation \KpiS\ indicates the total \kpi\ S-wave that is modelled by the LASS lineshape.}

It is possible to use the composition of the LASS lineshape to obtain separately the fractions of the contributing parts.
Integrating separately the resonant part, the effective range part, and the coherent sum, for both the \KpiSz and the \KpiSpm components,
the \KstarIIpm or \KstarIIzoptbar resonances are found to account for 78\%, the effective range term 46\%, and destructive interference between the two terms is responsible for the excess 24\%.
The branching fractions of the two nonresonant parts are found to be 
\begin{eqnarray*}
\Br{\Bs \to \KpiNRpm\Kmp}     &=& (11.4   \pm   0.8   \pm   0.2   \pm   9.2   \pm   1.2   \pm   0.5)  \times 10^{-6} \,, \\
\Br{\Bs \to \KpiNRz\KorKbarz} &=& (12.1   \pm   0.9   \pm   0.3   \pm   3.3   \pm   1.3   \pm   0.5)  \times 10^{-6} \,,
\end{eqnarray*}
where the fifth error is due to the uncertainty on the proportion of the \KpiS\ component due to the effective range part.
Similarly, the product branching fractions for the \KstarII\ resonances are
\begin{equation*}
\begin{array}{rcl}      
\Br{\Bs \to \KstarIIpm\Kmp; \KstarIIpm \to \KorKbarz\pipm}           &=& \\
& \multicolumn{2}{l}{\hspace{-14mm}(19.4   \pm   1.4   \pm   0.4   \pm    15.6   \pm   2.0   \pm   0.3)  \times 10^{-6} \,,} \\
\Br{\Bs \to \KstarIIzoptbar\KorKbarz; \KstarIIzoptbar \to \Kmp\pipm} &=& \\
& \multicolumn{2}{l}{\hspace{-14mm}(20.5   \pm   1.6   \pm   0.6   \pm \phz5.7   \pm   2.2   \pm   0.3)  \times 10^{-6} \,.}
\end{array}  
\end{equation*}
% where the fifth error is due to the uncertainty on the proportion of the \KpiS component due to the \KstarII resonance.

Results for the various \Kstar resonances are further corrected by their branching fractions to \kpi to obtain the quasi-two-body branching fractions.
The branching fractions to \kpi are~\cite{PDG2017}: $\Br{\KstarI \to \kpi} = 100\%$, $\Br{\KstarII \to \kpi} = (93 \pm 10)\%$ and $\Br{\KstarIII \to \kpi} = (49.9 \pm 1.2)\%$.
In addition, the values of $\Br{K^{*} \to K \pi}$ are scaled by the corresponding squared Clebsch-Gordan coefficients, 
\ie\ $2/3$ for both $\KstarzorKstarzb \to \Kpm \pimp$ and $\Kstarpm \to \KorKbarz \pipm$. 
The branching fractions are thus
%-------------------------------------------------------------------------------
\begin{eqnarray*}      
\Br{\Bs\to \KstarIpm\Kmp}              &=& (18.6      \pm   1.2   \pm   0.8   \pm   \phz4.0   \pm   2.0)  \times 10^{-6} \,, \\
\Br{\Bs\to \KstarIIpm\Kmp}             &=& (31.3      \pm   2.3   \pm   0.7   \pm      25.1   \pm   3.3)  \times 10^{-6} \,, \\
\Br{\Bs\to \KstarIIIpm\Kmp}            &=& (10.3      \pm   2.5   \pm   1.1   \pm      16.3   \pm   1.1)  \times 10^{-6} \,, \\
\Br{\Bs \to \KstarIzoptbar\KorKbarz}   &=& (19.8      \pm   2.8   \pm   1.2   \pm   \phz4.4   \pm   2.1)  \times 10^{-6} \,, \\
\Br{\Bs \to \KstarIIzoptbar\KorKbarz}  &=& (33.0      \pm   2.5   \pm   0.9   \pm   \phz9.1   \pm   3.5)  \times 10^{-6} \,, \\
\Br{\Bs \to \KstarIIIzoptbar\KorKbarz} &=& (16.8      \pm   4.5   \pm   1.7   \pm      21.2   \pm   1.8)  \times 10^{-6} \,,
\end{eqnarray*}  
%-------------------------------------------------------------------------------
where the uncertainties are respectively statistical, systematic related to experimental and model uncertainties, and due to
the uncertainty on ${\cal{B}}(\BsToKzBarOptKpi)$, $\Br{\Kstar\to\kpi}$ and,
in the case of \KstarII, the uncertainty of the proportion of the \KpiS
component due to the \KstarII resonance.

%The measurements of the previously observed decay modes
%$\Bs\to\KstarIpm\Kmp$ and $\Bs\to\KstarIzoptbar\KorKbarz$ are somewhat
%larger than the results reported in Refs.~\cite{LHCb-PAPER-2014-043}
%and~\cite{LHCb-PAPER-2015-018}, namely
%\begin{eqnarray}
%\Br{\Bs \to \KstarIpm\Kmp}           &=& \left( 12.7 \pm 1.9 \pm 1.9 \right) \times 10^{-6} \, , \\
%\Br{\Bs \to \KstarIzoptbar\KorKbarz} &=& \left( 10.9 \pm 2.5 \pm 1.2 \right) \times 10^{-6} \, .
%\end{eqnarray}
%Partly this is due to the increased BF of \BstoKsKPi from the updated
%analysis using both 2011 and 2012 data.
%Moreover, there are many improvements that have been brought to this analysis, starting from a new Stripping and 
%reconstruction to a more comprehensive signal model.
%Finally, this amplitude analysis can better separate the \KstarI states from the other contributions in the Dalitz plot, 
%in particular the S-wave.
%Therefore, the obtained results are meant to supersede the previous published measurement. 


%
\section{Summary}
\label{sec:summary}

In summary, the first amplitude analysis of $\Bs \to \KS\Kpm\pimp$ decays has been presented, using a data sample corresponding to $3.0\invfb$ of $pp$ collision data collected 
by the LHCb experiment. 
A good description of the data is obtained with a model containing contributions 
from both neutral and charged resonant states $\Kstar(892)$, $K^*_0(1430)$ and $K^*_0(1430)$. 
Measurements of the previously observed decay modes
$\Bs\to\KstarIpm\Kmp$ and $\Bs\to\KstarIzoptbar\KorKbarz$ are consistent with theoretical predictions~\cite{Cheng:2014uga,Li:2014fla,Li:2018qrm}, and also consistent with but larger than the previous LHCb results~\cite{LHCb-PAPER-2014-043,LHCb-PAPER-2015-018}, which they supersede.
This is attributed partly due to the larger \BstoKsKPi\ branching fraction determined in the updated analysis based on both 2011 and 2012 data~\cite{LHCb-PAPER-2017-010} compared to its previous determination~\cite{LHCb-PAPER-2013-042}. 
% and from improvements in the selection.
This amplitude analysis provides better separation of the \KstarI\ states from the other contributions in the Dalitz plot, in particular the S-wave, and more accurate estimation of associated systematic uncertainties.
% Therefore, the obtained results are meant to supersede the previous published measurement. 
Contributions from $K^*_0(1430)$ states are observed for the first time with significance above $10$ standard deviations. 

Increases in the data sample size will allow the reduction of both statistical and systematic uncertainties on these results.
As significantly larger samples are anticipated following the upgrade of LHCb~\cite{LHCb-TDR-012,LHCb-PII-EoI}, it will be possible to extend the analysis to include flavour tagging and decay-time-dependence and therefore to obtain sensitivity to test the SM through measurement of \CP\ violation parameters in $\Bs \to \KS\Kpm\pimp$ decays.




%
\section*{Acknowledgements}
%
% These Acknowledgements valid from 14-Aug-2018
%
\noindent We express our gratitude to our colleagues in the CERN
accelerator departments for the excellent performance of the LHC. We
thank the technical and administrative staff at the LHCb
institutes.
We acknowledge support from CERN and from the national agencies:
CAPES, CNPq, FAPERJ and FINEP (Brazil); 
MOST and NSFC (China); 
CNRS/IN2P3 (France); 
BMBF, DFG and MPG (Germany); 
INFN (Italy); 
NWO (Netherlands); 
MNiSW and NCN (Poland); 
MEN/IFA (Romania); 
MSHE (Russia); 
MinECo (Spain); 
SNSF and SER (Switzerland); 
NASU (Ukraine); 
STFC (United Kingdom); 
NSF (USA).
We acknowledge the computing resources that are provided by CERN, IN2P3
(France), KIT and DESY (Germany), INFN (Italy), SURF (Netherlands),
PIC (Spain), GridPP (United Kingdom), RRCKI and Yandex
LLC (Russia), CSCS (Switzerland), IFIN-HH (Romania), CBPF (Brazil),
PL-GRID (Poland) and OSC (USA).
We are indebted to the communities behind the multiple open-source
software packages on which we depend.
Individual groups or members have received support from
AvH Foundation (Germany);
EPLANET, Marie Sk\l{}odowska-Curie Actions and ERC (European Union);
ANR, Labex P2IO and OCEVU, and R\'{e}gion Auvergne-Rh\^{o}ne-Alpes (France);
Key Research Program of Frontier Sciences of CAS, CAS PIFI, and the Thousand Talents Program (China);
RFBR, RSF and Yandex LLC (Russia);
GVA, XuntaGal and GENCAT (Spain);
the Royal Society
and the Leverhulme Trust (United Kingdom);
Laboratory Directed Research and Development program of LANL (USA).


\addcontentsline{toc}{section}{References}
\setboolean{inbibliography}{true}
\bibliographystyle{LHCb}
\bibliography{main,LHCb-PAPER,LHCb-CONF,LHCb-DP,LHCb-TDR}
 
\newpage

% Author List ----------------------------
% You need to get a new author list!
%\input{LHCb_HD_authorlist_2014-06-20}
 
\end{document}
