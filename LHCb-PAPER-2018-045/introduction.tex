\section{Introduction}
\label{sec:Introduction}

The search for new sources of \CP violation in addition to that predicted by 
the CKM matrix~\cite{Cabibbo:1963yz,Kobayashi:1973fv} is among the main goals of current particle physics research.
One interesting approach is the study of decay-time distributions of neutral \B meson decays to hadronic final states mediated by the loop (``penguin'') $b \to s$ amplitude.  
As-yet undiscovered particles can contribute in the loops and cause the observables to deviate from their expected values 
in the Standard Model (SM)~\cite{Grossman:1996ke,Fleischer:1996bv,London:1997zk,Ciuchini:1997zp}.
Studies of various \Bz decays have been performed for this reason, including decay-time-dependent amplitude analyses of $B^0 \to \KS\pi^+\pi^-$~\cite{Dalseno:2008wwa,Aubert:2009me} and $B^0 \to \KS K^+K^-$~\cite{Nakahama:2010nj,Lees:2012kxa} transitions.
Such analyses, which involve describing the variation of the decay amplitudes over the full phase-space of the three-body decays, are more sensitive to interference effects than the ``quasi-two-body'' approach and hence are particularly important when broad resonances contribute.
Decay-time-dependent analyses of \Bs\ meson transitions mediated by hadronic $b \to s$ amplitudes have been performed for the $\Bs \to \Kp\Km$~\cite{LHCb-PAPER-2018-006}, $\Bs\to\phi\phi$~\cite{LHCb-PAPER-2014-026,LHCb-CONF-2018-001} and $\Bs\to\Kstarz\Kstarzb$~\cite{LHCb-PAPER-2017-048} decays, but not yet for any three-body \Bs\ decay.

The $\Bs\to\KS\Kpm\pimp$ channels have been observed~\cite{LHCb-PAPER-2013-042,LHCb-PAPER-2017-010}, and quasi-two-body measurements of the resonant contributions from $\Bs\to\Kstarpm\Kmp$~\cite{LHCb-PAPER-2014-043} and $\Bs\to\KstarzorKstarzb\KS$~\cite{LHCb-PAPER-2015-018} have also been performed.
These decays provide interesting long-term potential for time-dependent \CP\ violation measurements~\cite{Gronau:2006qn}.
The $\KS\Km\pip$ and $\KS\Kp\pim$ final states are both accessible to \Bs and \Bsb decays, with the corresponding amplitudes expected to be comparable in magnitude.
Example decay diagrams for contributions through the $\Bs \to \Kp\Kstarm~(\Kstarp\Km)$ and $\Bs \to \Kstarz\Kzb~(\Kz\Kstarzb)$ resonant processes are shown in Fig.~\ref{fig:feynman}, where subsequent transitions $\Kstarm \to \Kzb\pim$, $\Kstarz \to \Kp\pim$ and $\Kzb \to \KS$ and their conjugates lead to the $\KS\Kp\pim$ ($\KS\Km\pip$) final state for the former (latter) processes. 
The inclusion of charge conjugate processes is implied throughout the paper, except where explicitly stated otherwise.
Thus, large interference effects, and potentially large \CP\ violation effects, are anticipated.  
It is therefore of interest to study these channels with an amplitude analysis.

\begin{figure}[!tb]
  \centering
  \includegraphics*[width=0.46\textwidth]{figs/Generic_Bs2KstK_tree_ext}
  \includegraphics*[width=0.46\textwidth]{figs/Generic_Bs2KstK_loop}
  \includegraphics*[width=0.46\textwidth]{figs/Generic_Bs2KstK_loop_EW}
  \includegraphics*[width=0.46\textwidth]{figs/Generic_Bs2KsKz_loop}
  \caption{\small
    Feynman diagrams for (top left) external tree, (top right) internal penguin and 
    (bottom left) electroweak penguin contributions for $\Bs \to \Kp\Kstarm~(\Kstarp\Km)$ decays; and 
    (bottom right) internal penguin amplitude for the $\Bs \to \Kstarz\Kzb~(\Kz\Kstarzb)$ decay mode. 
    The electroweak penguin diagram for the $\Bs \to \Kstarz\Kzb~(\Kz\Kstarzb)$ channel is not shown; neither are diagrams corresponding to annihilation amplitudes. 
    In each case, the first set of final state particles (black) leads to the $\KS\Kp\pim$ final state, while the second set (blue) leads to $\KS\Km\pip$.
  }
  \label{fig:feynman}
\end{figure}
  
In this paper, the first Dalitz plot analysis of $\Bs\to\KS\Kpm\pimp$ decays is described.  
The analysis is based on a data sample corresponding to $3.0\invfb$ of $pp$ collision data recorded by the LHCb experiment during 2011 and 2012.
Due to the limited sample size available, the analysis is performed without considering decay-time dependence and without separating the $\Bs$ or $\Bsb$ initial states (\ie\ the analysis is ``untagged'').
Due to the modest effective tagging efficiency that can be achieved at hadron collider experiments, the inclusion of initial-state flavour tagging and decay-time dependence would not currently result in useful sensitivity to the additional parameters that can only be measured in such an analysis.
%Moreover, conservation of \CP\ violation in the decay amplitudes is assumed, in order to simplify the analysis by reducing the number of fitted parameters.

A novel feature of this analysis is that there are two independent final states ($\KS\Kp\pim$ and $\KS\Km\pip$) that are treated separately but simultaneously.
Denoting one by $f$ and the other by $\bar{f}$, then the former (latter) receives contributions from the amplitudes ${\cal{A}}_f$ and $\bar{\cal{A}}_f$ (${\cal{A}}_{\bar{f}}$ and $\bar{\cal{A}}_{\bar{f}}$), where ${\cal{A}}$ and $\bar{\cal{A}}$ are used to denote amplitudes for \Bs\ and \Bsb\ decays, respectively.
The untagged decay-time-integrated density of events in the Dalitz plot corresponding to $f$ therefore depends on $|{\cal A}_{f}|^2$ and $|\bar{{\cal A}}_{f}|^2$, while that for $\bar{f}$ depends on $|{\cal A}_{\bar{f}}|^2$ and $|\bar{{\cal A}}_{\bar{f}}|^2$.\footnote{
  The untagged decay-time-integrated rate also depends on an interference term that is responsible for the difference between the $t=0$ branching fraction and the decay-time-integrated branching fraction~\cite{DeBruyn:2012wj,LHCb-PAPER-2013-069,Dettori:2018bwt}.
  This must be considered when results are interpreted theoretically, but is not relevant for the discussion here.}
In the absence of \CP\ violation in decay ${\cal{A}}_f = {\bar{\cal{A}}}_{\bar{f}}$ and ${\bar{\cal{A}}}_f = {\cal{A}}_{\bar{f}}$, but there is no simple relation between ${\cal{A}}_f$ and ${\bar{\cal{A}}}_f$.
Indeed, theoretical predictions indicate that the values of these amplitudes could be quite different~\cite{Cheng:2014uga,Li:2014fla,Li:2018qrm}.
Thus, the situation differs from that usually considered in Dalitz plot analysis, where the density is given simply by the magnitude of a single amplitude squared. 

Precedent for handling this situation is taken from amplitude analyses of flavour-specific $B$ meson decays that do not account for \CP\ violation effects.  
In such analyses the distributions for $B$ and $\Bbar$ decays are added together, assuming them to be identical and fitting them with a single amplitude.
However, in the case that \CP\ violation effects are actually present, the true underlying density is given by the incoherent sum of two contributions, as here.  
In this example the fitted parameters of the amplitude model will differ from their true values by an amount that depends on the size of the \CP\ violation effects.
Similarly, by fitting each of the two $\Bs\to\KS\Kpm\pimp$ Dalitz plots with a single amplitude, the results will give values that differ from the true properties of the decays by amounts that must be estimated.
Detailed studies with simulated pseudo-experiments demonstrate that the fit fractions (defined in Sec.~\ref{sec:formalism}) obtained by this approach are biased by relatively small amounts that can be accounted for with systematic uncertainties, but that measurements of other quantities may not be reliable.
Therefore, the results of the analysis are presented in terms of fit fractions only.

The remainder of the paper is organised as follows.
In Sec.~\ref{sec:Detector}, a brief description of the LHCb detector, online selection algorithms and simulation software is given.
The selection of $\Bs\to\KS\Kpm\pimp$ candidates, and the method to estimate the signal and background yields are described in Sec.~\ref{sec:selection} and Sec.~\ref{sec:dataset}, respectively.
The analysis described in these chapters follows closely the methods used for the branching fraction measurement presented in Ref.~\cite{LHCb-PAPER-2017-010}.
As such, all four final states (\KsPiPi, \KsKpPim, \KsKmPip, and \KsKK, collectively referred to as $\Kshhp$ where $h$ represents either kaon or pion) are considered up to this point in the paper, but only $\KsKpPim$ and $\KsKmPip$ are discussed subsequently.
The Dalitz plot analysis formalism is given in Sec.~\ref{sec:formalism}, while inputs to the fit such as the signal efficiency and background distributions are contained in Sec.~\ref{sec:dalitz}.
Sources of systematic uncertainty are discussed in Sec.~\ref{sec:systematics} before the results are presented in Sec.~\ref{sec:results}.
A summary concludes the paper in Sec.~\ref{sec:summary}.

