\section{Detector, trigger and simulation}
\label{sec:Detector}

The \lhcb detector~\cite{Alves:2008zz,LHCb-DP-2014-002} is a single-arm
forward spectrometer covering the \mbox{pseudorapidity} range $2<\eta <5$,
designed for the study of particles containing \bquark or \cquark quarks.
The detector includes a high-precision tracking system consisting of a
silicon-strip vertex detector (\velo) surrounding the $pp$ interaction
region, a large-area silicon-strip detector located upstream of a dipole
magnet with a bending power of about $4{\mathrm{\,Tm}}$, and three stations
of silicon-strip detectors and straw drift tubes placed downstream of the
magnet.
The tracking system provides a measurement of momentum, \ptot, of charged particles with
relative uncertainty that varies from 0.5\% at low momentum to 1.0\% at 200\gevc.
The minimum distance of a track to a primary vertex (PV), the impact parameter (IP), 
is measured with a resolution of $(15+29/\pt)\mum$,
where \pt is the component of the momentum transverse to the beam, in\,\gevc.
Different types of charged hadrons are distinguished using information
from two ring-imaging Cherenkov detectors.
Photons, electrons and hadrons are identified by a calorimeter system
consisting of scintillating-pad and preshower detectors, an electromagnetic
calorimeter and a hadronic calorimeter.
Muons are identified by a system composed of alternating layers of iron and
multiwire proportional chambers.

The online event selection is performed by a
trigger~\cite{LHCb-DP-2012-004}, 
which consists of a hardware stage, based on information from the calorimeter and muon
systems, followed by a software stage, in which all charged particles
with $\pt>500\,(300)\mevc$ are reconstructed for data collected in 2011\,(2012).
At the hardware trigger stage, events are required to contain a muon with high
\pt or a hadron, photon or electron with high transverse energy in the
calorimeters.
The software trigger requires a two-, three- or four-track secondary vertex
with significant displacement from all primary $pp$ interaction vertices.
At least one charged particle must have transverse momentum $\pt >
1.7\,(1.6)\gevc$ in the 2011\,(2012) data and be inconsistent with
originating from a PV.
A multivariate algorithm~\cite{BBDT} is used for the identification of
secondary vertices consistent with the decay of a \bquark hadron.
It is required that the software trigger decision must have been caused
entirely by tracks from the decay of the signal \B candidate.

Simulated data samples are used to investigate backgrounds from other
\bquark-hadron decays and also to study the detection and reconstruction
efficiency of the signal.
In the simulation, $pp$ collisions are generated using
\pythia~\cite{Sjostrand:2007gs,*Sjostrand:2006za} with a specific \lhcb
configuration~\cite{LHCb-PROC-2010-056}.
Decays of hadronic particles are described by \evtgen~\cite{Lange:2001uf},
in which final-state radiation is generated using
\photos~\cite{Golonka:2005pn}.
The interaction of the generated particles with the detector, and its
response, are implemented using the \geant toolkit~\cite{Allison:2006ve,
*Agostinelli:2002hh} as described in Ref.~\cite{LHCb-PROC-2011-006}.

